%%%%%%%%%%%%%%%%%%%%%%%%%%%%%%%%%%%%%%%%%%%%%%%%%%%%%%%%%%%%%%%%%%%%%%%%%%%%%
%%%
%%% File: thesis.tex, version 0.1, May 2010
%%%
%%% =============================================
%%% This file contains a template that can be used with the package
%%% cs.sty and LaTeX2e to produce a thesis that meets the requirements
%%% of the Computer Science Department from the Technical University of Cluj-Napoca
%%%%%%%%%%%%%%%%%%%%%%%%%%%%%%%%%%%%%%%%%%%%%%%%%%%%%%%%%%%%%%%%%%%%%%%%%%%%%

\documentclass[12pt,a4paper,twoside,openright]{report}

\usepackage{cs}

% !!!!!!!!!!!!!!!!!!!!!!!!!!!! IMPORTANT NOTE !!!!!!!!!!!!!!!!!!!!!!!
% FOR THESIS IN ROMANIAN LANGUAGE COMMENT OUT THE FOLLOWING LINE !!!!
\usepackage[english,romanian]{babel}


\graphicspath{{figures/}}
\ifpdf
  \DeclareGraphicsExtensions{.pdf,.jpeg,.png}
\else
  \DeclareGraphicsExtensions{.eps}
\fi

% \mastersthesis
\diplomathesis
% \leftchapter
% \centerchapter
% \rightchapter
\singlespace
% \oneandhalfspace
% \doublespace

\renewcommand{\thesisauthor}{Răzvan Teslaru}    %% Your name.
\renewcommand{\thesismonth}{Februarie}     %% Your month of graduation.
\renewcommand{\thesisyear}{2016}      %% Your year of graduation.
\renewcommand{\thesistitle}{Modul VNC pentru acces de la distanță la buffer-ul video al unui calculator prin interfața AMT} %{VNC Python Extension}

\renewcommand{\thesissupervisorname}{S.l.Dr.Ing. Adrian Coleșa}


%\renewcommand{\thesisdedication}{To my beloved wife and parents}

\begin{document}


% ================================================================================
% ======================== FIRST TITLE PAGE =====================================
% ================================================================================

\begin{titlepage}

\thispagestyle{firststylewithoutfooter}

% \fancyhead{}
% \fancyhead[C]{\includegraphics[width=20pt]{header-utcn-engleza.png}}
% \chead{\includegraphics[width=20pt]{header-utcn-engleza.png}}


\begin{center}
% {\scshape \universitynameenglish} \\
% {\scshape \facultynameenglish} \\
% {\scshape \depa1rtmentnameenglish} \\

% {\scshape \universitynameromanian} \\
{\scshape \facultynameromanian} \\
{\scshape \departmentnameromanian} \\


\vspace{6cm}

\thesistitlesize {\textbf{\thesistitle}\\}
\vspace {1cm}

%\thesistypesize \textbf{\thesistypeenglish}\\

\thesistypesize \textbf{\thesistyperomanian}\\


\vspace{2cm}

% \thesisauthortypesize \thesisauthortypeenglish \\ \textbf{\thesisauthor} \\
\thesisauthortypesize \thesisauthortyperomanian \\ \textbf{\thesisauthor} \\

\vspace{1cm}

% \thesissupervisorsize \thesissupervisorenglish \\ \textbf{\thesissupervisorname}\\
\thesissupervisorsize \thesissupervisorromanian \\ \textbf{\thesissupervisorname}\\



\vspace{\stretch{1}}
{\thesismonth} {\thesisyear} \\
\end{center}
\end{titlepage}

\begin{titlepage}
\phantom{1}
\end{titlepage}


% ================================================================================
% ======================== SECOND TITLE PAGE =====================================
% ================================================================================

\begin{titlepage}

\begin{center}

\thispagestyle{firststylewithfooter}

% {\scshape \universitynameenglish} \\
% {\scshape \facultynameenglish} \\
% {\scshape \departmentnameenglish} \\

% {\scshape \universitynameromanian} \\
{\scshape \facultynameromanian} \\
{\scshape \departmentnameromanian} \\

\vspace{1cm}

\newcolumntype{R}{>{\raggedleft\arraybackslash}X}%
\begin{tabularx}{\textwidth}{lR}
% {\scshape \facultydeanenglish} & {\scshape \deptmanagerenglish} \\
{\scshape \facultydeanromanian} & {\scshape \deptmanagerromanian} \\
\facultydeanname & \deptmanagername\\
\end{tabularx}

\vspace {2cm}

\thesistitlesize {\textbf{\thesistitle}\\}
\vspace {1cm}

% \thesistypesize \textbf{\thesistypeenglish}\\
\Large \textbf{\thesistyperomanian}\\

\vspace{1cm}

\end{center}

% \vspace{1cm}

\begin{flushleft}
\begin{enumerate}
%   \item \textbf{\thesisauthortypeenglish}: \thesisauthor
  \item \thesisauthortyperomanian: \thesisauthor
 
%  \item \textbf{\thesissupervisorenglish}: \thesissupervisorname
 \item \thesissupervisorromanian: \thesissupervisorname
 
%  \item \textbf{\thesiscontentsenglish}: Thesis presentation, suprvisior evaluation, chapter 1, chapter 2, \dots, chapter n, References, Anexes, CD.
 \item \thesiscontentsromanian: Pagina de prezentare, aprecierile coordonatorului, titlul capitolului 1, titlul capitolului 2, \dots, titlul capitolului n, bibliografie, anexe, CD.
 
%  \item \textbf{\thesisworkingplaceenglish}: UTCN, Cluj-Napoca
 \item \thesisworkingplaceromanian: UTCN, Cluj-Napoca

%  \item \textbf{\thesisadvisorsenglish}: Donald Knuth, Leslie Lamport, others \dots
% \item \thesisadvisorsromanian: Donald Knuth, Leslie Lamport, others \dots

%  \item \textbf{\thesisbegindateenglish}: \dotfill
 \item \thesisbegindateromanian: \dotfill

%  \item \textbf{\thesisenddateenglish}: \dotfill
 \item \thesisenddateromanian: \dotfill

\end{enumerate}

\end{flushleft}

\vspace{0.5cm}

\begin{center}

\newcolumntype{R}{>{\raggedleft\arraybackslash}X}%
\begin{tabularx}{\textwidth}{lR}
% {\thesissignatureenglish} {\thesissupervisorenglish} & {\thesissignatureenglish} {\thesisauthortypeenglish} \\
{\thesissignatureromanian} {\thesissupervisorromanian} & {\thesissignatureromanian} {\thesisauthortyperomanian} \\
\thesissupervisorname & \thesisauthor \\
\end{tabularx}

\vspace{\stretch{1}}
{\thesismonth} {\thesisyear} \\

\end{center}

\end{titlepage}


\begin{titlepage}
\phantom{1}
\end{titlepage}


% ================================================================================
% ======================== THIRD TITLE PAGE =====================================
% ================================================================================

\begin{titlepage}

\begin{center}
\thispagestyle{firststylewithfooter}

% {\scshape \universitynameenglish} \\
% {\scshape \facultynameenglish} \\
% {\scshape \departmentnameenglish} \\

% {\scshape \universitynameromanian} \\
{\scshape \facultynameromanian} \\
{\scshape \departmentnameromanian} \\
\end{center}

\vspace{3cm}

\begin{center}
% \autheticitydeclarationsize \textbf{\autheticitydeclarationenglish}
\autheticitydeclarationsize \textbf{\autheticitydeclarationromanian}
\end{center}

\vspace{1cm}

Subsemnatul \textit{\thesisauthor}, legitimat cu \textit{CI} seria \textit{XC} numărul \textit{666468}, CNP \textit{1910917045353}, autorul lucrării \textit{\thesistitle} elaborată în vederea susținerii examenului de finalizare a studiilor de masterat la Facultatea de Automatică și Calculatoare, Departamentul Calculatoare, Specializarea \textit{Calculatoare} din cadrul Universității Tehnice din Cluj-Napoca, sesiunea \textit{\thesismonth} a anului universitar \textit{2016/2017}, declar pe proprie răspundere, că această lucrare este rezultatul propriei mele activități intelectuale, pe baza cercetărilor mele și pe baza informațiilor obținute din surse care au fost citate în textul lucrării și în bibliografie.

Declar că această lucrare nu conține porțiuni plagiate, iar sursele bibliografice au fost folosite cu respectarea legislației române și a convențiilor internaționale privind drepturile de autor.

Declar, de asemenea, că această lucrare  nu a mai fost prezentată în fața unei alte comisii de examen de licență sau disertație.

În cazul constatării ulterioare a unor declarații false, voi suporta sancțiunile administrative, respectiv, \textit{anularea examenului de licență}.


\vspace{2cm}

\begin{center}

\newcolumntype{R}{>{\raggedleft\arraybackslash}X}%
\begin{tabularx}{\textwidth}{lR}
% Cluj-Napoca & {\thesissignatureenglish{ }\thesisauthortypeenglish}\\
% date  & {\thesisauthor} \\
Cluj-Napoca & {\thesissignatureromanian}\\
data  & {\thesisauthortyperomanian} \\ 
\end{tabularx}

\end{center}


\end{titlepage}


\begin{titlepage}
\phantom{1}
\end{titlepage}


%\pagestyle{headings}

% ABSTRACT
\begin{abstract}
%Descrierea sumară a lucrării, în câteva fraze. Un site foarte util ce conține exemple \LaTeX se găsește la \url{http://en.wikibooks.org/wiki/LaTeX}.
%Recomandăm crearea unui proiect în editoare Latex specializate (exemplu Kile pe Linux, TexnicCenter pe Windows) astfel încât compilarea/translatarea codului Latex în PDF să se facă din orice fișier editați la un moment dat (practic se va compila proiectul). De asemenea, dacă fișierul ``thesis.bib'' este inclus în proiect, există facilitatea de code-completion și la cite. 

Un server de testare a cărui mașini client rulează constant poate să reprezinte o sarcină suplimentară ce necesită timp și oameni pentru întreținere, sau poate să reprezinte un sistem complex cu capacități de mentenanță la distantă și funcționalități de întreținere automatizată. Această lucrare își propune să prezinte modul în care am abordat problema accesului la distanță la buffer-ul video a mașinilor prin intermediul Active Management Technology.
\end{abstract}

\begin{titlepage}
\phantom{1}
\end{titlepage}



%\thesistitle                    %% Generate the title page.
%\authordeclarationpage                %% Generate the declaration page.

\pagenumbering{roman}
\setcounter{page}{1}

\tableofcontents
\newpage

\listoftables

\listoffigures

\begin{titlepage}
\phantom{1}
\end{titlepage}

% \clearpage
\newpage

\pagenumbering{arabic}
\setcounter{page}{1}

\pagestyle{normalpagestyle}
\renewcommand{\chaptermark}[1]{ \markboth{\thechapter. #1}{} }
\renewcommand{\sectionmark}[1]{ \markright{\thesection. #1}{} }



%\chapter{Introduction}
\chapter{Introducere}
\label{cap:Introducere}


%Ce se scrie aici:
%\begin{itemize}
%    \item Contextul
%    \item Conturarea/Descrierea domeniului exact al temei
%    \item Se răspunde la întrebările: \textbf{ce} (s-a făcut)?, \textbf{de ce} (s-a făcut, adică motivația; ce se rezolvă, la ce este util, etc.)?, \textbf{cum} (s-a făcut, adică particularitățile abordării, prezentate sumar).
%    \item Introducerea se termină cu o descriere a conținutului lucrării, de genul: Cap X descrie ..., Cap Y prezintă ...
%    \item Introducerea reprezintă o sinteză a lucrării, din care cititorul trebuie să-și poată da bine seama dacă lucrarea prezintă sau nu interes pentru el. 
%    \item Se poate organiza pe subsecțiuni, dacă se dorește, după exemplul de mai jos, dar nu e obligatoriu asta, având în vedere dimensiunea mică
%    \item reprezintă cca 5\% din lucrare (nu mai mult de 2-4 pagini)
%\end{itemize}

\section{Context}

%Despre contextul în care este abordată și se aplică tema lucrării.
Industria IT este într-o dezvoltare accelerată, dar la fel ca oricare alt domeniu are ramuri care sunt suprapopulate și ramuri care necesită îmbunătățite. Creșterea amenințărilor și atacurilor cibernetice generează o cerere foarte mare pentru securitate, atât în cadrul utilizatorilor obișnuiți cât și al companiilor. Pentru a putea oferi soluții capabile să protejeze împotriva oricărui fel de risc și vulnerabilitate, sistemele de securitate trebuie să aibă în primul rând o calitate înaltă. O astfel de performanță este dobândită printr-o testare amănunțită.

În cadrul unei companii se folosesc diverse unelte hardware, software și metodologii de testare. Există o varietate mare de unelte dedicate testării automate, de performanță sau chiar de tracking, dar de cele mai multe ori acestea nu sunt suficient de diversificate sau optimizabile datorită caracterului generic care îl îmbracă, și se optează pentru dezvoltarea de soluții inhouse.

Este cu atât mai dificilă testarea cu cât proiectele sunt mai îndrăznețe. Dorința de eficientizare a procesului de debugging a declanșat necesitatea unei infrastructuri de testare cât mai complexă. Un asemenea sistem necesită o multitudine de feature-uri cu cât mai puține dependințe externe, pentru a furniza un acces low-level în diverse faze de execuție.


%\section{Motivation}
\section{Motivație}
%De ce este utilă abordarea temei? Ce probleme rezolvă și ce rezultate poate aduce?
Un avantaj extraordinar îl constituie abilitatea unui server de testare de a se conecta la distanță la un calculator client pentru a vedea starea în care se află. Pentru a soluționa această problemă putem să o abordăm în două moduri: folosirea unei surse existente sau implementarea unui protocol de la zero. Deoarece implementarea unui protocol ar fi mult prea costisitoare, iar utilizarea software-urilor existente au dezavantajul că nu expun buffer-ul video, s-a optat pentru utilizarea unei biblioteci. Cea mai apropiată soluție de nevoile noastre are un dezavantaj: arhitectura a fost gândită în așa fel încât o singură conexiune este posibilă la un moment dat. Astfel provocarea s-a transformat în scrierea unui wrapper pentru această bibliotecă, care să poată depăși acest impediment.

Inițial proiectul a fost gândit ca o temă de cercetare. S-a cerut efectuarea unui studiu pe tehnologii hardware și firmware AMT(Active Management Technology). Acestea au o multitudine de funcționalități: management la distanță, mentenanță, posibilitate de update și upgrade. După se analizează potențialul și se stabilesc mai multe ținte, se încearcă realizarea unu prototip pe unul din ele. Având în vedere că analiza s-a realizat cu mult timp înainte, s-a ales ca subiect de interes capacitatea de conexiune remote pentru a dispune direct de Pixel Buffer. 

După ce se realizează un prototip minimalist și funcțional în limbajul C, se va încerca realizarea unui software similar, dar scris pentru utilizare în limbajul Python. Pentru implementarea în Python s-a ales varianta scrierii unui wrapper peste librăria C găsită(vncamt.dll), în detrimentul rescrierii tuturor protocoalelor de comunicare dintre server și client folosind AMT. Trebuie luat în considerare că s-a cerut să fie dezvoltat in Python 3, lucru care va îngreuna sarcina deoarece s-au produs numeroase schimbări de la versiunea anterioară, și multe dintre modulele necesare nu au fost încă actualizate pentru versiunea de Windows. Nu este importantă metoda de redare grafică. Obiectivul principal este capacitatea de a oferi acces direct la buffer-ul video într-o manieră paralelă.

%\section{Report's Structure}
\section{Structura lucrării}
Capitolul~\ref{cap:obiective-specificatii} prezintă atât obiectivele proiectului de licență cât și obiectivele viitoare.\\
Capitolul~\ref{cap:studiu-bibliografic} descrie sursele biografice utilizate în procesul de dezvoltare al acestui proiect. Deciziile de dezvoltare au fost influențate și prin compararea cu alte soluții existente. Sunt incluse detalii despre tehnologiile și algoritmii folosiți.\\
Capitolul~\ref{cap:fund-teoretice} are ca obiectiv fixarea tuturor noțiunilor folosite în proiect. Sunt prezentate numeroase concepte relevante temei..\\
Capitolul~\ref{cap:analiza-si-proiectare} include descrieri detaliate în legătură cu design-ul proiectului: arhitectura generală și detaliată a sistemului, diagramă de stare, diagrame de secvențe, prezentarea algoritmilor, plus o serie de avantaje și dezavantaje.\\
În capitolul~\ref{cap:implementare} se prezintă organizarea codului, clasele și API-urile importante, descrierea algoritmilor principali.\\
Capitolul ~\ref{cap:rezultate} are ca obiectiv prezentarea metodelor de validare a soluțiilor și interpretări de performanță.
Capitolul ~\ref{cap:user-manual} descrie pașii de parcurs pentru a efectua instalarea și rularea extensiei.\\
Capitolul ~\ref{cap:concluzii} prezintă un rezumat al contribuțiilor aduse și o descrierea a posibilelor dezvoltări și îmbunătățiri.


%\chapter{Project's Objectives and Specification}
\chapter{Obiective și specificații}
\label{cap:obiective-specificatii}

%Acest capitol conține descrierea detaliată a temei de cercetare propriu-zise, formulată exact, cu obiective clare și specificații, pe 2-3 pagini și eventuale figuri explicative. Titlul nu e neapărat impus și, de asemenea, capitolul poate fi inclus ca subcapitol în Capitolul~\ref{cap:Introducere}, dacă se potrivește.
%
%Reprezintă cca. 5--10\% din lucrare.

Așa cum am menționat și în capitolul anterior, unul din primele obiective a fost o temă de cercetare în scopul găsirii de tehnologii și funcționalități oferite de Intel AMT care se dovedesc utile în diverse aplicații. În urma acestor analize s-a hotărât un studiu mai amănunțit pe caracteristicile de KVM(Keyboard Video Mouse). SDK-ul de Intel AMT conține foarte multe exemple și biblioteci utile. Inițial s-a încercat realizarea unu proof of concept în limbajul C folosind librăria vncamt.dll. Acest mic proiect a avut ca scop principal familiarizarea cu librăria: documentație, arhitectură, structuri.

După etapa de setup și configurare a mașinilor client folosite, s-a încercat utilizarea unor funcții elementare folosind modul default de desenare. Următorul pas a fost acela de realiza un viewer custom. Aceasta a fost o necesitate doar, pentru a demonstra ca toate callback-urile destinate actualizării buffer-ului video sunt implementate corect și în același timp funcționale.

Mai departe s-a cerut realizarea unui wrapper în Python pentru această bibliotecă. S-a ales limbajul Python deoarece în acest limbaj se dorește scrierea unui sistem mai mare. Wrapper-ul reprezintă doar o componentă și trebuie să fie utilizabil în diverse scenarii. Acest proiect are ca scop principal furnizarea accesului remote la buffere video simultan, lucru care oferă o gamă foarte largă de posibilități.

%\section{Objectives} 
\section{Obiective}

%Obiectivele proiectului sunt lucrurile care se dorește a fi realizate, ca urmare a abordării temei lucrării de licență. În general numărul de obiective este proporțional cu timpul de care dispunem. Exemple generice:
%\begin{enumerate}
%  \item Analiza critică a soluțiilor existente pentru problema abordată și identificarea posibile limitări ale acestora.
%  \item Propunerea unor soluții la (o parte) din problemele identificate. 
%  \item Implementarea soluției și validarea ei
%  \item Identificarea unor teme de dezvoltare și cercetare ulterioare
%  \item \dots
%\end{enumerate}

În această secțiune vom puncta ideile majore ale proiectului, inclusiv viitoare posibilități oferite de acestă soluție.
Printre obiectivele inițiale și probabil și obiectivele viitoare, tema de cercetare include următoarele puncte generice:
\begin{enumerate}
  \item Analiza critică a soluțiilor existente și a posibilităților oferite de Intel AMT.
  \item Control asupra opțiunilor din BIOS.
  \item Bootare de pe orice sursă, MBR sau UEFI.
  \item Capturare de video stream live.
  \item Capturare de screenshot-uri,
  \item Manipularea stării de alimentare: power on, shut down, sleep, reboot, hibernate.
  \item Logare de informații low level.
  \item Interogare de software / hardware.
  \item Serial prin rețea.
\end{enumerate}

După ce s-a hotărât abordarea unei soluții în Python pentru realizarea unei conexiuni VNC, obiectivele au fost urmatoarele:
\begin{enumerate}
  \item Analiza soluțiilor existente și identificarea posibile limitări ale acestora.
  \item Propunerea unor soluții la (o parte din) problemele identificate. 
  \item Implementarea soluției și validarea ei.
  \item Identificarea unor teme de dezvoltare și cercetare ulterioare.
\end{enumerate}

%\section{Project Specification} 
\section{Specificații}
În continuare se prezintă o descriere amplă despre caracteristicile proiectului de licență. Specificațiile tehnice ale soft-ului sunt o expunere a cerințelor funcționale și a cerințelor non-funcționale. Acestea au rolul de a pune bazele în privința a ceea ce se așteaptă de la proiect, cât și la ce nu se așteaptă.

%\subsection{Functional Specification}
\subsection{Specificații funcționale}

%Aplicația realizată:
%\begin{itemize}
%  \item va face următoarele ...
%  \item va oferi următoarea funcționalitate \dots
%  \item va afișa o o interfață
%  \item va fi bazată pe modelul \dots (client-server) 
%  \item va fi implementată în C, Java etc.
%  \item \dots
%\end{itemize}
Cerințele funcționale nu definesc felul în care se realizează implementarea. Ele descriu modul în care agenții vor interacționa cu produsul. De cele mai multe ori reprezintă un punct de plecare și îndrumare pentru programatori. Aceste cerințe sunt date de ceea ce se dorește să se obțină la finalul dezvoltării temei.

Aplicația realizată:
\begin{itemize}
  \item va folosi librăria vncamt.dll
  \item va fi implementată în Python
  \item va afișa o interfață grafică, dar este flexibilă din acest punct de vedere
  \item va expune o interfață simplă prin intermediul căruia componenta se poate integra ușor într-un proiect mai mare
  \item va realiza conexiuni cu mai multe calculatoare simultan
  \item va oferi posibilitatea de extindere a funcțiilor
  \item oferă posibilitatea agenților să se conecteze la mașini folosind doar IP-ul de AMT
  
%  \item va implementa   
%  \item va face următoarele ...
%  \item va oferi următoarea funcționalitate \dots  
%  \item va fi bazată pe modelul \dots (client-server) 
\end{itemize}

%\subsection{Non-Functional Specification}
\subsection{Specificații non-funcționale}

%Aplicația trebuie, de asemenea, să aibă următoarele caracteristici non-funcționale (exemple):
%\begin{itemize}
%  \item să aibă următoarea performanță
%  \item să fie ușor/intuitiv de utilizat
%  \item să fie adoptată pe scară largă 

%  \item o extensie C pentru Python, deoarece modulul ctypes nu a fost suficient pentru a realiza wrapper-ul direct in Python
%  \item \dots
%\end{itemize}


Majoritatea rezultă din constrângerile incluse în specificația cerințelor utilizatorilor. Cerințe de: performanță, interfață, de operare, de verificare, de portabilitate, de întreținere, de fiabilitate. Sunt atașate cerințelor funcționale.
\begin{itemize}
  \item Managementul erorilor:\\
Această cerință are scopul de a ascunde eșecurile interne și de a preveni datele invalide introduse de către utilizator. Pentru realizarea cerinței s–a folosit afișarea de mesaje corespunzătoare pentru a–l avertiza pe user despre incorectitudinea datelor introduse.

  \item Performanță:\\
Se referă la memoria necesară pentru rulare și timpul de execuție necesar pentru finalizarea operației efectuate. Pentru o performanță cât mai ridicată s–au folosit librării care oferă soluții eficiente pentru operații cu tipuri de variabile mari.

  \item Testabilitate:\\
Reprezintă numărul de erori apărute în timpul rulării și modul în care a fost testată aplicația.
Pentru a testa aplicația în timpul dezvoltării s–au folosit încercări manuale de decodificare a datelor primite. Pentru determinarea corectitudinii s-au făcut teste pe o imagine care există o singură culoare primară, de exemplu roșu. Astfel s–a putut compara cu tupla RGB decodificată din informația primită.

  \item Utilizabilitate:\\
Se referă la complexitatea utilizării aplicației din punct de vedere al interfeței  oferite de către modul.
Pentru o complexitate cât mai scăzută s–a ales implementarea unei interfețe simple și sugestive astfel încât utilizatorul să nu întâmpine probleme în integrare.
\end{itemize}

%\chapter{Bibliographic Survey}
\chapter{Studiu bibliografic}
\label{cap:studiu-bibliografic}

%Documentarea bibliografică are ca obiectiv fixarea referențialului în care se situează tema, prezentarea susrselor bibliografice utilizate și a cercetărilor similare și raportarea abordării din lucrare la acestea.
%
%Referințele bibliografice se vor face pentru fiecare carte, articol sau material folosit pentru elaborarea lucrării de licență. 
%
%Reprezintă cca. 10--15\% din lucrare. -> 5.5--8.5pg


%\section{Related Work}
\section{Abordări similare}

%Comparați abordarea voastră cu cele ale altor soluții: ce e asemănător, ce e diferit (și, de preferat, mai bun). 
%
%Citarea referințelor se face ca în exemplele \ref{subsec:s10} din Bibliografie. 
%Vezi citările următoare.
%
%În articolul \cite{Antoniou04} autorul descrie configurația tehnică a unei "honeynet" și prezintă câteva atacuri de actualitate asupra honeynet, precum și o serie de recomandări pentru securizarea sistemelor conectate la rețele de calculatoare.
%
%% În capitolul 4 al [], referitor la valoare honeypots, Spitzner prezintă avantajele și dezavantajele acestora.
%
%În articolul on-line \cite{electronic-citation} găsim detalii interesante despre \dots.

La momentul actual există o multitudine de soluții software care oferă într-un fel sau altul acces la un calculator aflat la distanță. Deși scopul principal este același, găsim diferențe semnificante de la o aplicație la alta. Se pot distinge numeroase abordări, de la modul în care user-ul realizează conexiunea până la nivelul pachetelor de date și modul în care diverse protocoale aleg să realizeze această comunicare remote.

În continuare sunt prezentate cele mai folosite soluții existente, reprezentative pentru protocolul implementat.


\subsection{TeamViewer}
%wiki si site oficial
Unul din cele mai cunoscute produse existente pe piață este TeamViewer. Cu acesta vă puteți conecta la orice PC sau server din lume în câteva secunde. Este practic un pachet de programe care oferă diverse funcționalități adiționale pe lângă posibilitatea de partajare a desktop-ului: conferințe web și chiar transfer de fișiere între calculatoare. 

Conexiunea se realizează în urma unei scurte instalări. Ambele calculatoare trebuie să ruleze aplicația. Când TeamViewer este pornit, generează o parolă și un ID. Conexiunea se poate realiza în ambele direcții: clientul local are nevoie de ID-ul și parola clientului remote pentru a avea acces asupra clientului remote, și invers.

%http://stackoverflow.com/questions/9498877/how-is-teamviewer-so-fast
La baza TeamViewer implementează Real-Time Transport Protocol, folosit pentru transferul audio și video%[]wikiRTP.
Este cel mai des utilizat pentru streaming: telefonie, conferințe video, servicii de televiziune. Este doar unul din secretele acestui software extrem de rapid. Este preferat de cei care au nevoie să lucreze la distanță într-un mod cât mai eficient. Protocolul folosit de proiectul nostru, Virtual Network Computing(VNC), este unul bazat pe pixeli. %wiki realvnc
Deși este mai puțin eficient în comparație cu protocoalele ce trimit modificările primitivelor grafice sau comenzi high-level într-un format simplu(ex: open window), VNC este mai flexibil și poate să afișeze orice tip de desktop.

\subsection{Remote Desktop Connection}
Remote Desktop Connection este o aplicație client a RDS. Oferă unui utilizator posibilitatea să se logheze remote la un calculator aflat în rețea, ce rulează serviciul de server. Această soluție folosește Remote Desktop Protocol(RDP), un protocol de proprietate dezvoltat de Microsoft. Prin convenție, acesta folosește portul TCP 3389 ți portul UDP 3389.%wiki RDP si RDC 
Există clienți pentru diverse versiuni de Microsoft Windows, pentru Windows Mobile, Linux, Unix, iOS, Android și alte sisteme de operare.

Deși este rapid și acoperă o gamă largă de platforme, acest protocol este privat. Obiectivul acestui proiect, așa cum a fost menționat și în capitolul anterior, este să aibă acces complet la buffer-ul video. 

\subsection{RealVNC}
%translate from wiki-- https://en.wikipedia.org/wiki/RealVNC
RealVNC este o companie care oferă un produs software pentru acces remote. Programul constă într-o aplicație client-server destinată să controleze  un alt calculator de la distanță folosind protocolul Virtual Network Computing(VNC). Acesta poate rula pe diverse sisteme de operare, cum ar fi: Windows, Mac, mai multe distribuții de Unix. În același timp, clienții pot rula pe diverse platforme: Java, device-uri Apple și Android. Singurul client exclusiv Windows este cel inclus în VNC Viewer Plus, conceput pentru a interfața cu serverul încorporat pe chipset-urile Intel AMT ce se găsesc pe plăcile de bază Intel vPro.

Acest produs este cel mai similar cu proiectul deoarece folosim aceeași librărie pentru a realiza comunicarea prin intermediul Intel AMT. Este folosit protocolul Remote FrameBuffer(RFB). Conexiunea se realizează în mod implicit pe portul TCP 5900. Spre deosebire de RealVNC, în momentul de față proiectul nu oferă o gamă atât de variată de implementări încât să poată suporta toate feature-urile sau platformele care le oferă RealVNC, dar având în vedere că nici nu este nevoie în momentul de față să avem aceste opțiuni, nu reprezintă o problemă. 
În schimb soluția propusă de noi oferă posibilitatea conectării la mai multe viewere în același moment. Restul funcționalităților se pot adăuga pe parcurs.


%\section{Technologies and Methods}
\section{Tehnici/Tehnologii/Surse folosite}

%Sursele de documentare referitoare la metodele, tehnologiile, ideile folosite. 

Capitolul 2 din \cite{carte-amt} menționează că administrarea, ca o disciplină unica, a evoluat istoric din nevoia tot mai mare de a configura și menține sistemele informatice , aplicațiile , precum și utilizarea rețelelor . Utilizarea de instrumente de administrare la distanță a devenit importantă. Acest lucru a condus la dezvoltarea de mai multe protocoale pentru administrarea de la distanță. Industria a început să lucreze la standarde interoperabile care permit sistemelor mai multor producători să fie gestionate remote cu instrumente comune.

\subsection{Intel vPro}

Resursa \cite{site:intel-vpro} prezintă Intel vPro tehnology ca o colecție de înaltă performanță, eficientă din punct de vedere energetic, precum și capabilități robuste de management care permit profesioniștilor să monitorizeze, să gestioneze, și să repare calculatoare de la distanță, indiferent de starea sistemului de operare sau de starea de alimentare a calculatorului. Aceleași tehnici care au fost folosite pentru a reduce costurile de proprietate și consumului de energie, precum și pentru a crește fiabilitatea și securitatea serverelor mari, sunt construite în platforme cu tehnologia Intel vPro.

Hardware-ul sistemului poate lua parte la autentificare prin rețea, chiar înainte de pornirea sistemului de operare, permițând administratorului să aibă acces în orice moment.


\subsection{Intel Active Management Technology}

Aplicațiile software care folosesc Intel AMT  \cite{site:intel-amt} sunt o altă parte importantă a gestionării platformelor cu tehnologia Intel vPro. Capacitățile Intel vPro Technology sunt expuse utilizează interfețe bazate pe standarde care să permită interoperabilitatea ușor și integrarea cu o varietate largă de soluții de administrare și de securitate.

Intel Active Management Technology este o componentă majoră a platformei Intel vPro ce o diferențiază de alte platforme. În continuare vom prezenta  caracteristicile cheie. 
Intel AMT oferă o serie de caracteristici care permit descoperirea, vindecare, și protejarea platformei și a resurselor. Aceste capacități pot fi accesate cu ajutorul interfețe locale sau de rețea într-o manieră sigură.

Hardware-ul de Intel AMT, așa cum este ilustrat și în figura \ref{amt} este format din chipset-ul Intel , care include controlerul de rețea Ethernet, cipul comunicare wirelss, memorie flash nevolatilă pentru codul și stocarea datelor, precum și legăturile de comunicare. Firware conține mediu runtime, kernel, drivere, și servicii firmware.

%%%%%%%%%%%%%% ARHITECTURA AMT
\begin{figure}
    \centering
    \includegraphics[width=0.5\textwidth]{amt}
    \caption{High Level View of Intel AMT components}
    \label{amt}
\end{figure}

Niciuna dintre caracteristicile Intel Active Management Technology nu ar fi de folos, dacă acestea nu ar putea fi combinate pentru a rezolva probleme din lumea reală. În capitolul 6 din cartea \cite{carte-amt} este prezentat un set de scenarii reale și cum caracteristicile Intel AMT pot fi utilizate pentru a le rezolva. Este de asemenea important de reținut faptul că în cele mai multe din aceste scenarii, soluțiile software pot să nu funcționeze deloc sau nu rezolvă problema la fel de eficient.

%final
Intel AMT trebuie să lucreze cu infrastructurile de întreprindere existente, la fel ca mai multe dintre platformele Intel vPro, ce sunt folosite într-un mediu IT dotat standard. Pentru a facilita această integrare, Intel AMT oferă o serie de capabilități pentru a face integrarea în medii enterprise simplă. Cele mai multe dintre aceste componente nu sunt necesare în medii mai mici. Intel AMT SDK conține documentația pentru interfațarea cu Intel AMT, împreună cu bibliotecile necesare, și oferă de asemenea codul sursă al unor exemple.


\subsection{Virtual Network Computing}

\begin{figure}
    \centering
    \includegraphics[width=0.4\textwidth]{vnc}
    \caption{Sistem VNC}
    \label{vnc}
\end{figure}


Intel a făcut un lucru semnificativ prin colaborare cu furnizori independenți de software. RealVNC este o companie care oferă o soluție de acces de la distanță. Software-ul constă dintr-o aplicație server și client pentru protocolul Virtual Network Computing pentru a controla ecranul altui calculator remote. 

Principiile VNC, evidențiate în articolul \cite{vnc} și în figura \ref{vnc}, sunt menționate în \cite{vnc-guide} după cum urmează:

\begin{itemize} 
  \item Serverul VNC este programul care împărtășește ecranul său. Serverul permite pasiv clientului să preia controlul asupra acestuia.
  \item Clientul VNC, sau telespectatorul, este programul care vede, controlează, și interacționează cu serverul. Clientul controlează serverul.
  \item Protocolul VNC, protocol Remote Framebuffer(RFB), este foarte simplu, bazat pe primitive grafice de la server la client (Pune un dreptunghi de datele de pixeli la poziția specificată \cite{rfb}) și mesajele de eveniment de la client la server.
\end{itemize}

În interiorul Intel ME, figura \ref{real-vnc}, se găsește un server ce permite RealVNC să vadă de la distanță ceea ce se întâmplă pe ecranul computerului și de a controla mouse-ul și tastatura. VNC este un standard pentru portul 5900, pe care datele sunt transmise fără nici o prelucrare. Standardul de VNC nu implică criptare. 

VNC are unele limitări în protocolul care îl folosește. RFB funcționează prin transmiterea dreptunghiurilor de pixeli într-o rețea. Cu cât rezoluția și adâncimea de biți este mai mare cu atât este necesară o lățimea de bandă mai mare pentru a trimite actualizări. Există câteva optimizări VNC servere / clienți deja făcute: trimite numai regiunile modificate, regiunile neschimbate rămân stocate în cache-ul clientului.

%http://vpro.by/intel-amt-kvm-vnutrennee-ustroystvo-i-nastroyka
\begin{figure}
    \centering
    \includegraphics[width=0.85\textwidth]{real_vnc}
    \caption{RealVNC și IntelME}
    \label{real-vnc}
\end{figure}

\subsection{Keyboard Video Mouse}

Un switch KVM este un dispozitiv hardware care permite unui utilizator să controleze mai multe computere de la una sau mai multe seturi de tastaturi, monitoare video, și mouse-uri. Deși mai multe computere sunt conectate la KVM, de obicei un număr mai mic de calculatoare pot fi controlate la un moment dat. Dispozitivele moderne au adăugat posibilitatea de a partaja alte periferice cum ar fi dispozitive USB și audio.

Dispozitive de partajare KVM în comparație cu KVM switch sunt invers; adică un singur PC poate fi conectat la mai multe monitoare, tastaturi, și mouse-uri. Deși nu la fel de comune, această configurație este utilă atunci când operatorul dorește să acceseze un singur calculator de la două sau mai multe locații - de exemplu, un birou acasă cu un computer.

Suportul KVM de la Intel AMT a venit de la o versiune Intel AMT 6. Este un server built-in Intel ME RealVNC. Acest produs thid-party, disponibil în Intel ME 6.x și versiuni mai noi, permite oricărui client VNC compatibil să gestioneze computere bazate pe AMT. Un server RealVNC, aflat în arhitectura AMT este folosit pentru conexiunea la porturi pentru a oferi acces KVM Intel AMT.

\subsection{Remote Framebuffer}

Virtual Network Computing este un sistem grafic de partajare desktop care utilizează protocolul Remote Framebuffer pentru a controla de la distanță alt calculator. Acesta transmite  evenimente de tastatură și mouse de la un computer la altul, de actualizare de ecran înapoi în cealaltă direcție, printr-o rețea. 

RealVNC utilizează protocolul RFB, figura \ref{rfb-protocol}. Aceasta funcționează implicit pe portul TCP 5900. Când se face o conexiune prin Internet, utilizatorul trebuie să se deschidă acest port în firewall. Ca alternativă, VNC poate prin SSH  să stabilească o conexiune, evitând deschiderea porturilor suplimentare. De asemenea SSH oferă  posibilitatea de criptare a conexiunii dintre serverul VNC și client.

În \cite{wiki:vnc} se menționează ca RFB nu este în mod implicit un protocol securizat. În timp ce parolele nu sunt expuse, o tentativă de cracking s-ar putea dovedi de succes dacă ambele, cheia de criptare și parola codificată, sunt furate dintr-o rețea. Din acest motiv se recomandă să se utilizeze o parolă de cel puțin 8 caractere. Pe de altă parte, există o limită de 8 caractere pe unele versiuni ale VNC; în cazul în care se trimite o parolă de peste 8 caractere, caracterele în plus sunt eliminate și șirul trunchiat este comparat cu parola.

În cazul în care un client se deconectează de la un server dat și se reconectează ulterior la același server, starea este păstrată. Mai mult decât atât, un client diferit poate fi utilizat pentru a se conecta la același
server de RFB. Utilizatorul va vedea exact aceeași interfață grafică ca utilizatorul inițial. Astfel, interfața la aplicațiile utilizatorului devine complet mobilă. Ori de câte ori există o conexiune de rețea adecvată, utilizator își poate accesa propriile aplicații personale, iar starea acestor aplicații este
conservată între accese din diferite locații. 

%rfb protocol
\begin{figure}
    \centering
    \includegraphics[width=0.5\textwidth]{rfb_prot}
    \caption{Remote Framebuffer Protocol}
    \label{rfb-protocol}
\end{figure}

Inputul protocolului se bazează pe un model de workstation standard: tastatură
și dispozitiv de indicare cu mai multe butoane(exemplu mouse). Evenimentele de intrare sunt pur și simplu trimise la server de către client ori de câte ori utilizatorul apasă un buton sau ori de câte ori se modifică poziția pointerul dispozitivului. Aceste evenimente de intrare pot fi, de asemenea, primite de la alte dispozitive de intrare / ieșire non-standard. De exemplu ,un motor recunoașterea scrisului de mână pe bază de stilou, ar putea genera evenimente tastatură.


\subsection{Librăria vncamt}

VNC de fapt se comportă mai bine în comparație cu alte protocoale și funcționează pe aproape orice platformă.\\

Biblioteca de vizualizare expune un API de C. Este un DLL closed-source furnizat de RealVNC, implementarea unui viewer utilizând protocolul RFB 4.0. Această bibliotecă acceptă numai RFB pe bază de autentificare prin parolă. DLL suportă afișarea grafică în oricare dintre cele două metode:

\begin{itemize}
  \item Graficile sunt realizate de către viewer într-o fereastră separată în mod implicit.
  \item Graficile sunt realizate de către aplicația apelantă în propria fereastră GUI. Aplicația comunică cu biblioteca prin intermediul callback-urilor, primește informațiile pixelilor de la platforma remote, și să deseneze în propria fereastră, pe baza informațiilor despre pixeli.
\end{itemize}

Layer-ul de transport stabilește o conexiune TCP cu serverul KVM prin intermediul unei adrese IP și portului. Conexiunea poate fi realizată prin utilizarea proxy KVM.\\

În \cite{amt-sdk} se menționează că SDK-ul este limitat la un singur vizualizator  VNC ce poate exista în condiții de siguranță, fiecare cu procesul său. Nu trebuie încercată crearea unui al doilea vizualizator VNC în cadrul aceluiași proces, fără ca anterior să fi distrus prima instanță. Există posibilitatea de a porni și opri un singur vizualizator VNC de mai multe ori, de exemplu, pentru a încerca să se reconecteze la același server după ce o parolă a fost introdusă greșit.

În \cite{realvnc-sdk} se prezintă toată librăria și conține instrucțiuni de folosire.  Există două moduri de a folosi SDK-ul. Cel mai simplu mod este de a permite desenarea să se realizeze într-o fereastră desktop. Această fereastră afișează conținutul desktop al serverului VNC și folosește tastatură și mouse-ul pentru a genera evenimente de intrare pentru server. Utilizând fereastra desktop implicită necesită foarte puțin cod, dar nu oferă aproape nicio capacitate asupra interfeței utilizatorul. Nu este posibilă adăugarea unui meniu sau bară de instrumente la fereastra desktop.

Dacă nu doriți să utilizați fereastra desktop implicită, vă puteți înregistra callback-urile, astfel încât SDK vă poate anunța când s-au făcut modificări la server. Acum aplicația trebuie să deseneze buffer-ul video al serverului. De asemenea, SDK-ul nu poate colecta evenimente de intrare de la tastatură și mouse, așa că va trebui să fie interceptate și furnizate și acestea. Cu toate acestea, va oferi libertatea de a implementa și integra un viewer VNC în aproape orice aplicație, precum și pentru a adăuga orice caracteristici interfeței utilizatorilor.


%\chapter{Theoretical Backgound}
\chapter{Fundamente teoretice}
\label{cap:fund-teoretice}

%Aici se descriu pe scurt aspecte teoretice pe care se bazează lucrarea. Conținutul acestui capitol trebuie gândit pentru un citor care nu e specializat pe domeniul temei și nu cunoaște chestiunile de bază despre subiect. Pentru un cititor specializat, capitolul poate să stabilească un limbaj comun, relativ la termenii care pot fi interpretați diferit. 
%
%Acest capitol nu trebuie gândit și scris nici ca un copy-paste din alte surse, nici ca zona de reglaj a numărului de pagini ale lucrării. Deși va conține chestiuni pe care le-ați studiat și voi și pe care v-ați bazat, el trebuie să fie o compilare a surselor folosite, care să aibă sens și relevanță pentru lucrarea voastră. Trebuie să fie o descriere coerentă și logică a unor aspecte care ușurează sau fac posibilă înțelegerea părților următoare ale lucrării. Nu trebuie intrat insă prea mult în detalii, ci spuse doar chestiunile esențiale. 
%
%Dacă preluați text, figuri, tabela etc. din sursele de documentare, acestea din urmă trebuie indicate explicit. 
%
%Reprezintă cca. 10--15\% din lucrare. --> 5.5--8.25pg

\section{Protocolul Remote Framebuffer}
Așa cum se menționeză în \cite{rfb}, Remote Framebuffer este un protocol simplu pentru accesul de la distanță la interfața grafică a utilizatorului. Deoarece funcționează la nivelul framebuffer este aplicabil tuturor sistemelor și aplicațiilor, inclusiv Windows și Macintosh. RFB este
protocolul utilizat în Virtual Network Computing.

La prima vedere acest protocol pare să folosească o modalitate ineficientă de a desena mai multe componente grafice. Cu toate acestea, permițând codificări diferite pentru datele pixelilor, ne oferă un grad mare de flexibilitate, viteza de desenare a clientului și viteza de procesare a serverului.

O secvență de dreptunghiuri constituie o actualizare framebuffer. O actualizare reprezintă o schimbare de la o stare valabilă la alta a framebuffer-ului; în unele moduri este similar cu un cadru de film. 

\subsection{Codificare}
Interacțiunea inițială între client și serverul RFB implică o negociere a formatului și codificarea cu care vor fi trimise datele de pixeli. Această negociere a fost conceput pentru a face munca clientului cât mai ușoră. Ideea este că serverul trebuie să fie întotdeauna în măsură să furnizeze datele de pixeli în forma în care clientul dorește. Cu toate acestea, dacă clientul este capabil să facă față în mod egal cu mai multe formate sau codificări diferite, se poate alege una care este mai ușor pentru server. Tipurile de codificare definite în prezent sunt Raw, CopyRect , RRE , Hextile și ZRLE.

Formatul în pixeli se referă la reprezentarea culorilor individuale folosind valorile pixelilor. Cele mai comune formate pixel sunt de 24 de biți sau 16 biți " true color ". Există doua metode: ori valoarea pixelilor se traduce direct în intensități de culoare roșie, verde și albastră, ori se folosește un color map ce poate fi utilizat să traducă din valorile pixelilor în intensități RGB.

\subsection{Transmiterea de mesaje}
Tipul PIXEL reprezintă o valoare de a bytesPerPixel octeți, unde 8 x bytesPerPixel este numărul de biți pe pixel, așa cum a fost convenit de către client și server, fie în mesajul ServerInit sau în mesajul SetPixelFormat.

Odată ce clientul și serverul sunt siguri că o să vorbească unul cu altul, protocolul trece la faza de inițializare. Clientul trimite un mesaj ClientInit urmat de către serverul de trimiterea unui mesaj ServerInit.

După primirea mesajului ClientInit, serverul transmite un mesaj ServerInit. Acesta spune clientul lățimea și înălțimea buffer-ului serverului, formatul său de pixeli și numele desktop-ului asociat. Formatul pixel este prezentat în figura \ref{pixell}.

\begin{figure}
    \centering
    \includegraphics[width=0.5\textwidth]{pixel}
    \caption{Pixel Format}
    \label{pixell}
\end{figure}

\subsection{Actualizare Framebuffer}
Notifică serverul că un client este interesat de zona de framebuffer specificat de  poziția x, y, lățime și înălțime. Serverul de obicei raspunde la un FramebufferUpdateRequest prin trimiterea unei FramebufferUpdate. Un singur FramebufferUpdate poate fi trimis ca răspuns la mai multe FramebufferUpdateRequests.

Serverul presupune că clientul păstrează o copie a tuturor părților frame buffer-ului. Acest lucru înseamnă că, în mod normal serverul are nevoie doar să trimită actualizări elementare clientului.

În cazul în care clientul nu a pierdut nici un conținut, atunci trimite un FramebufferUpdateRequest cu un set incremental diferit de zero(true). Dacă și când au loc schimbări Atunci când vor exista schimbări ale frame buffer-ului, serverul va trimite un FramebufferUpdate. Între FramebufferUpdateRequest
și FramebufferUpdate există o perioadă nedeterminată de timp.

Un update de framebuffer constă dintr-o secvență de dreptunghiuri, ca în figura \ref{fbu}, de date de pixel care clientul ar trebui să le pună în framebuffer-ul său. Acesta este trimis ca răspuns la FramebufferUpdateRequest din partea clientului. 

\begin{figure}
    \centering
    \includegraphics[width=0.5\textwidth]{fbu}
    \caption{Update frame buffer }
    \label{fbu}
\end{figure}	

Aceasta este urmată de un număr de dreptunghiuri ce conțin date despre pixeli în forma \textit{numberOfRectangles}, visibil în \ref{nr}

\begin{figure}
    \centering
    \includegraphics[width=0.5\textwidth]{nr}
    \caption{\textit{numberOfRectangles}}
    \label{nr}
\end{figure}	

\subsection{Raw Pixel Data}

Cel mai simplu tip de codificare este raw pixel data. În acest caz, datele constau din înălțimea x lățimea pixelilor, unde lățimea și înălțimea sunt cele ale dreptunghiului. Valorile reprezintă pur și simplu fiecare pixel, în ordine de la stânga la dreapta precum scanline. Toți clienții RFB trebuie să fie capabil să decodifice datele de pixeli în acest sistem de codare raw pixel data, iar serverele RFB ar trebui să producă numai codare raw, cu excepția cazului în care clientul solicită în mod special pentru un alt
tip de codare.



\section{Tehnologii}


%\subsection{Librăria RealVNC: vncamt.dll}

\subsection{Numpy}

Extensiile Python sunt deosebit de simplu de înțeles, deoarece toate au o structură foarte asemănătoare. Desigur, NumPy nu este o extensie banală în Python, și poate lua un pic mai mult de înțeles. NumPy oferă un API de C pentru a permite utilizatorilor să extindă sistemul și să obțină accesul la obiect matrice pentru a fi utilizate în alte rutine. NumPy este pachetul fundamental pentru calcul științific în Python. Acesta conține, printre altele :

\begin{itemize}
  \item un obiect matrice puternic N -dimensionale
  \item funcții(de broadcasting) sofisticate
  \item instrumente pentru integrarea C / C ++ și codul Fortran
  \item algebra liniară, transformări Fourier, și capabilități de generare numere aleatoare
\end{itemize}

Pe lângă utilizările sale evidente științifice, NumPy poate fi utilizat și ca un eficient container multidimensional a datelor generice. Pot fi definite tipuri de date arbitrare. Acest lucru permite integrarea NumPy perfect și rapid.

Așa cum este specificat în capitolul \textit{Python Types and C-Structures} din \cite{numpy}, tipurile Python sunt echivalentul funcțional în C al clase în Python. Construirea unui nou tip Python pune la dispoziție un nou obiect de Python. Obiectul ndarray este un exemplu de un tip nou definit în C. Tipurile noi sunt definite în C prin două etape de bază:
\begin{enumerate}
  \item Crearea unei structuri C, numită de obicei Py{Nume}Object, care este compatibilă cu structura PyObject, dar deține și informațiile suplimentare necesare acestui obiect;
  \item Popularea tabelului PyTypeObject cu pointeri la funcțiile care implementează comportamentul dorit pentru acest tip.
\end{enumerate} 

Este posibil să se creeze array-uri multidimensionale în NumPy. Desigur, array-urile nu sunt limitate la o singură dimensiune în NumPy. Ele sunt de dimensiuni arbitrare. Noi le crea prin formarea de liste,sau tuple, imbricate. Asignarea și accesarea elementelor unei matrice este similar cu alte tipuri de date secvențiale din Python, adică liste și tupluri. Avem mai multe opțiuni pentru indexare, ceea ce face ca indexarea în NumPy este o caracteristică foarte puternică și similare core Python.


\subsection{Pygame}

Pygame este un set de module Python de cross-platform concepute pentru scrierea de jocuri video. Acesta include grafica pe calculator și biblioteci de sunet concepute pentru a fi utilizate cu limbajul de programare Python. Acesta este construit peste biblioteca Simple Directmedia Layer, cu intenția de a permite în timp real dezvoltare de joc pe calculator fără mecanica de nivel scăzut a limbajului de programare C și a derivatelor sale. Se bazează pe presupunerea că cele mai costisitoare funcții din interiorul jocurilor pot fi separate de logica de joc, ceea ce face posibil să se utilizeze un limbaj de programare de nivel înalt, cum ar fi Python.

Pygame a fost construit pentru a înlocui PySDL după ce dezvoltarea acestuia a stagnat. Pygame fost scris inițial de Pete Shinners și este distribuit sub licența open source free software publică GNU. Acesta a fost un proiect comun începând din 2004 sau 2005. PySDL2 este un wrapper al bibliotecii SDL2 și este similară cu proiectul întrerupt PySDL.

Pygame și SDL sunt un engine de C excelent pentru jocuri 2D. Cea mai mare parte a execuției se petrece in interiorul SDL. SDL pot profita de accelerarea grafică hardware. Permițând acest lucru un joc poate rula în jurul valorii de 200 de cadre pe secundă(fps) în loc de 40 de cadre pe secundă

Acest modul, surfacearray, pygame poate utiliza pachetele externe de tip NumPy sau pachete Numeric. Conține funcții pentru conversia datele pixelilor din suprafețele pygame în matrice, și invers.

Fiecare pixel este stocat ca o singură valoare întreagă pentru a reprezenta culorile roșu, verde și albastru. Imaginile 8 bit utilizează o valoare care apare într-un colormap. Pixeli cu adâncime mai mare utilizează un proces de împachetare pentru a plasa trei sau patru valori într-un singur număr.

Matricele sunt indexate pe axa X, urmată de axa Y. Matricea care tratează pixelii ca un singur număr întreg sunt denumite array-uri 2D. Acest modul poate separa valorile de culoare roșie, verde și albastru în indici separați. Aceste tipuri de matrice sunt denumite array-uri 3D, iar ultimul indice este 0 pentru roșu, 1 pentru verde și 2 pentru albastru.

\subsection{Python C Api}

Api-ul este coloana vertebrală a interpretorului standard Python, altfel cunoscut ca CPython. Folosind acest API este posibil să se scrie modulul de extensie Python în C și C ++. Evident, aceste module de extensie pot să apeleze orice funcție scrisă în C sau C ++.La fel și NumPy vine cu o extensie implementată în C. Acest API poate fi folosit pentru a crea și manipula matrice NumPy din C, atunci când scriem extensii C. 

Printre avantaje se numără urmatoarele: nu necesită biblioteci suplimentare, o control low level, în întregime utilizabil în C ++. Dar are și dezavantaje:
poate necesita o cantitate substanțială de efort, trebuie să fie compilate, cost ridicat de întreținere, nu are forward compatibility datorită modificărilor de API C, bug-urile de numărare de referință sunt ușor de creat și foarte greu de găsit.

Există două motive fundamentale pentru utilizarea API-ului Python / C. Primul motiv este de a scrie module de extensie pentru scopuri specifice; acestea sunt module C, care extind interpretorul Python. Aceasta este probabil cel mai comună utilizare. Al doilea motiv este de a folosi Python ca o componentă într-o aplicație mai mare; această tehnică este în general referită ca încorporarea Python într-o aplicație.

Cele mai multe funcții ale API-ului au una sau mai multe argumente, precum și o valoare returnată, de tip \textit{PyObject *}. Acest tip este un pointer la un tip de date opac și reprezintă un obiect Python arbitrar. Toate obiectele Python, chiar întregi Python, au un tip și un număr de referință. Tipul unui obiect determină ce fel de obiect este.

Există câteva alte tipuri de date care joacă un rol important în API. Cele mai multe sunt tipuri simple de C, cum ar fi \textit{int}, \textit{long}, \textit{double} și \textit{char}. Câteva tipuri de structuri sunt utilizate pentru a descrie tabelele statice, utilizate pentru a lista funcțiile exportate de un modul sau atributele date ale unui nou tip de obiect. Alt tip de structură este folosit pentru a descrie valoarea unui număr complex.

Alte modalități de a interfața cod C cu Python: 

\begin{enumerate}
  \item Modulul \textbf{Ctypes} din Python este probabil cel mai simplu mod de a apela funcțiile C din Python. Modulul ctypes oferă tipuri de date compatibile cu limbajul C și funcții pentru a încărca DLL-uri, astfel încât apelarea metodelor poate fi făcută pentru bibliotecilor C, fără a fi nevoie să le modifice. Faptul că partea de C nu trebuie atinsă, adaugă la simplitatea acestei metode. Acesta poate fi folosit pentru a crea un wrapper peste biblioteci, exclusiv în Python.

  \item Simplified Wrapper and Interface Generator(\textbf{SWIG}) este un alt mod de a interfața cod C în Python. În această metodă, dezvoltatorul trebuie să dezvolte un fișier de interfață în plus, care este o intrare pentru SWIG. Dezvoltatorii python în general nu folosesc această metodă deoarece este în cele mai multe cazuri inutil de complexă. Aceasta este o metodă buna atunci când vrem să interfațăm cod C / C ++ cu mai multe limbi diferite.
  
Cel mai important lucru cu SWIG este că poate genera automat codul wrapper-ului. În timp ce acesta este un avantaj în ceea ce privește timpul de dezvoltare, poate fi de asemenea o povară. Fișierul generat tinde să fie destul de mare. iar nivelurile multiple de indirectare care sunt un rezultat al procesului de wraping, îl pot face dificil de înțeles.

  \item \textbf{Cython} este atât un limbaj asemanător Python pentru scrierea de extensii C cât și un compilator avansat. Limbajul Cython este un superset al Python, care vine cu construcții suplimentare ce permit apelul de funcții C, și adnotează variabile de clasă și atribute cu tipuri C. În acest sens îl putem numi un Python cu tipuri.
  
În timp ce alte soluții care auto generează cod pot fi destul de dificil de depanat, de exemplu SWIG, Cython vine cu o extensie pentru debugger GNU care ajută la depanarea Python, Cython și C.
\end{enumerate}


În \cite{} se oferă principalele motive pentru care se foloseste API-ul de C: viteză, C este de aproximativ 50 de ori mai rapid decât Python; bibliotecile C funcționează foarte bine, astfel încât nu se dorește rescrierea lor în Python; acces la anumite resurse de nivel scăzut.

După cum se poate vedea în \cite{capi}, o extensie Python C ce folosește API-ul trebuie să includă \textit{Python.h}. Această metodă pare complexă la început, dar odată înțeleasă se poate dovedi a fi destul de folositoare.



\subsection{Compilare} 

Pentru a compila toate acestea putem folosi \textbf{distutils}. Trebuie să ne asigurăm că includem anteturile NumPy. Sistemul standard de compilare Python oferă suport pentru extensii C, ceea ce este destul de convenabil. Se salvează urmatoarea secvența de cod în fișierul \textit{setup.py}:

\lstset{language=Python,frame=single, showstringspaces=false}
\begin{lstlisting}
from distutils.core import setup, Extension
setup(name='<nume_modul>', version='1.0', \
ext_modules=[Extension('<nume_modul>', ['<nume_modul>.c'])])
\end{lstlisting}

Așa cum este specificat în lucrarea \cite{py-book}, fișierul poate conține diverse informații cum ar fi: numele modulului, versiunea, o listă cu toate pachetele care dorim să le includem, un dicționar care să specifice fișierele necesare fiecărui pachet.

După ce se rulează comanda \textit{python setup.py install},  se buildează și instalează fișierul C în modulul Python. Pentru a testa că operația s-a finalizat cu succes, se importă modulul și se apelează. Se generază mai multe tipuri de fișiere:
\begin{itemize}
  \item \textbf{.pyc}: Acesta este bytecode-ul compilat. Dacă se importată un modul, Python va construi un fișier .pyc care conține bytecode-ul pentru a face mai ușor și mai rapid  importul acestuia din nou mai târziu.
  \item \textbf{.pyo}: Acesta este un fișier .pyc care a fost creat în timp flag-ul de optimizări a fost activat. Singurul lucru care este mai rapid la fișierele \textit{.pyo} sau \textit{.pyc} este viteza cu care sunt încărcate.
  \item \textbf{.pyd}: Acesta este de fapt un scipt Python realizat ca un DLL de Windows.
\end{itemize}

Există uneori cazuri în care se dorește dezvoltarea de programe care rulează atât pe versiunile de Python 2, cât și pe cele de Python 3. Există o mulțime de diferențe între acestea, inclus în cazul scrierii unei extensii. Pentru această problemă există doua abordări. În primul caz se poate realiza și distribui două module, pentru Python 2 și 3. Cealaltă opțiune este să modificăm codul curent astfel încât să fie compatibil cu orice versiune.

%\subsection*{•}


\section{Log și debugging}

Aproape în fiecare limbaj de programare, manipularea erorilor este o activitate dificilă. C-ul abordează problema returnând coduri de eroare și setând valoare variabilei globale \textit{errno}. Pe măsură ce se scrie cod C, se folosește modelul:

\begin{itemize}
  \item Se apelează o funcție.
  \item În cazul în care valoarea returnată este o eroare, trebuie verificată de fiecare dată.
  \item Apoi se face cleanup-ul tuturor resurselor create până în prezent.
  \item Se afișează un mesaj de eroare.
\end{itemize}

Soluția folosită este un mic set de macro-uri care pun în aplicare un sistem de bază de depanare pentru C. Acest sistem este ușor de înțeles, funcționează cu orice bibliotecă, și face codul mai mai clar.

Directivele \textit{ifdef} verifică dacă token-ul dat a fost definit mai devreme în fișier sau într-un fișier inclus. Dacă este așa, acesta include tot ceea ce există pană la linia \textit{else}, sau dacă aceasta nu există, până la linia \textit{endif}. În soluția propusă am definit doi tokeni: \textit{logging} și \textit{debug}. Împreună definesc comportamentul mai multor macro-uri. Există trei tipuri de logare: mesaj, warning și eroare; plus un tip debug, fiecare cu trăsăturile sale specifice:

\begin{itemize}
  \item Log Message: Afișează mesajul dorit cu tag-ul \textit{[info]} în față.
  \item Log Warning: Afișează mesajul dorit cu tag-ul \textit{[warn]} în față folosind culoarea galben. Are în plus informații despre fișierul C și linia de unde a fost apelat macro-ul.
  \item Log Error: Afișează mesajul dorit cu tag-ul \textit{[error]} în față folosind culoarea roșu. Acest macro are rezervat un argument în plus, special pentru codul de eroare.
  \item Debug: Este cel mai uzual. Oferă și informații despre fișierul și linia de unde a fost apelat. Foarte util în momentul în care nu volumul de mesaje de printare este mare, devenind astfel mai eficient să dezactivăm toate macro-urile odată în loc să le ștergem sau să le comentăm pe toate.
\end{itemize}
%\chapter{Analysis and Design}
\chapter{Analiză și proiectare}
\label{cap:analiza-si-proiectare}

%Acest capitol descrie design-ul proiectului și cuprinde, în general: 
%\begin{enumerate}
%  \item ilustrarea arhitecturii generale și detaliate a sistemului implementat, care să evidențieze modulele componente și relațiile dintre acestea
%  \item stările prin care trece sistemul în decursul funcționării sale (diagrame de stare)
%  \item modul de interacțiune dintre module și funcționalitatea acestora ilustrată prin diagrame de secvențe
%  \item descrierea algoritmilor/metodelor pe care se bazează funcționarea sistemului dezvoltat
%  \item descrierea organizării/structurii eventualelor baze de date folosite
%  \item justificarea alegerilor/deciziilor făcute și analiza critică a acestora (avantaje și dezavantaje), prin comparație cu alte alternative posibile
%\end{enumerate}
%
%Ca idee generală, design-ul trebuie să fie prezentat independent de o implementare anume, în general, și de cea a voastră, în particular. De asemenea, descrierea design-ului trebuie să conțină toate elementele și detaliile necesare, astfel încât altcineva decât voi să poate realiza o implementare a lui, fără a fi nevoit să ia decizii arhitecturale sau organizare (adică, de design) și să vă contacteze pentru a-și lămuri anumite aspecte neclare.
%
%Capitolul trebuie organizat pe secțiuni și subsecțiuni astfel descrierea să urmeze un cors logic și ușor de urmărit. 
%
%Ponderea acestui capitol relativ la întreaga lucrare este de 25-35\%.
%
%
%\section{Examples: lists, figures, tables, equations}
%
%Așa arată o listă de elemente nenumerotate:
%\begin{itemize}
%  \item element 1
%  \item element 2
%  \item \dots
%\end{itemize}
%
%
%Așa arată o listă de elemente numerotare:
%\begin{itemize}
%  \item element 1
%  \item element 2
%  \item \dots
%\end{itemize}
%
%
%Așa arată o listă în text: 
%\begin{inparaenum}[(\itshape 1 \upshape)]
%  \item element 1, 
%  \item element 2, 
%  \item \dots
%\end{inparaenum}
%
%\textbf{Atenție}: orice tabel, figura sau ecuație (formulă) trebuie referite \textit{explicit} în text explicit (de genul: în Figura X este ulustrat \dots, în Tabelul Y se poate vedea \dots), pentru că Latex le poate plasa chiar și pe altă pagină decât acolo unde vrem noi să ne referim la ele. Vedeți exemple de mai jos!
%
%Tabelul~\ref{table:example} ilustrează un exemplu de tabel. Un editor on-line de tabele poate fi găsit la \url{http://www.tablesgenerator.com/}. 
%
%\begin{table}[t]
%\centering                          % tabel centrat 
%\begin{tabular}{|c|c|c|c|}          % 4 coloane centrate 
%\hline\hline                        % linie orizontala dubla
%Case & Method\#1 & Method\#2 & Method\#3 \\ [0.5ex]   % inserare tabel
%%heading
%\hline                              % linie orizontal simpla
%1 & 50 & 837 & 970 \\               % corpul tabelului 
%2 & 47 & 877 & 230 \\
%3 & 31 & 25 & 415 \\[1ex]           % [1ex] adds vertical space
%\hline                              
%\end{tabular}
%\caption{Nonlinear Model Results}   % titlul tabelului
%\label{table:example}                % \label{table:nonlin} introduce eticheta folosita pentru referirea tabelului in text; referirea in text se va face cu \ref{table:nonlin}
%\end{table}
%
%În Figura~\ref{fig:exemplu} 
%
%\begin{figure}
%    \centering
%    \includegraphics[width=0.5\textwidth]{image}
%    \caption{Numele figurii}
%    \label{fig:exemplu}
%\end{figure}
%
%
%Formula~(\ref{eq:example}) arată modul de calcul al lui $\Delta$:
%\begin{equation} \label{eq:example}
%    \Delta =\sum_{i=1}^N w_i (x_i - \bar{x})^2 .
%\end{equation}
%
%
%Algoritmul~\ref{alg:example} este un exemplu de descriere pseudo-cod a unui algoritm, preluat de la \href{http://en.wikibooks.org/wiki/LaTeX/Algorithms#Typesetting_using_the_algorithm2e_package}{http://en.wikibooks.org/wiki/LaTeX}. El utilizează pachetul \textit{algorithm2e}. Alternativ, puteți utiliza pachetele \textit{algorithmic} sau \textit{program}. 
%
%\begin{algorithm}
% \KwData{this text}
% \KwResult{how to write algorithm with \LaTeX2e }
% initialization\;
% \While{not at end of this document}{
%  read current\;
%  \eIf{understand}{
%   go to next section\;
%   current section becomes this one\;
%   }{
%   go back to the beginning of current section\;
%  }
% }
% \caption{How to write algorithms}
% \label{alg:example}
%\end{algorithm}
%
%25-35% --> 13.75-19.25pg

O aplicație de Virtual Network Computing constă din două tipuri de componente. Un server, care generează un display, și un viewer, care se ocupă de desenarea acestui display pe ecran. Există două caracteristici importante:



\begin{itemize}
  \item Serverul și viewer-ul se pot afla pe calculatoare diferite, care au chiar și arhitecturi diferite. Protocolul care realizează conexiunea dintre server și viewer este simplu, este disponibil pentru toată lumea, și nu este dependent de platformă.
  \item Viewer-ul nu salvează nicio stare. În cazul în care se întrerupe conexiunea cu serverul și se reia ulterior, după reconectare nu există niciun fel de pierderi de date. Din această cauză conexiunea se poate realiza dintr-o altă locație, oferind astfel mobilitate.
\end{itemize}

\begin{figure}
    \centering
    \includegraphics[width=0.7\textwidth]{clientserver}
    \caption{\textit{Client și server VNC}}
    \label{vncclsrv}
\end{figure}

În figura \ref{vncclsrv} este evidențiată comunicarea. Aceasta este cu adevărat un protocol thin-client. A fost conceput pentru a face foarte puține cerințe ale viewer-ului. În acest fel, clienții pot rula pe o gamă variată de hardware, iar sarcina de a implementa un client este făcută cât mai simplu posibil.

Protocolul VNC este un protocol simplu pentru accesul de la distanță la interfețe grafice. Se bazează pe conceptul de Remote Framebuffer. În trecut protocolul VNC era referit ca protocol RFB, astfel încât este posibil să fi întâmpinat acest termen în alte publicații. Protocolul permite pur și simplu unui server să actualizeze framebuffer-ul afișat de un vizualizator. Deoarece funcționează la nivel de framebuffer este aplicabil tuturor sistemelor de operare. Protocolul va funcționa folosind orice tip transport de încredere, cum ar fi TCP / IP.

\section{Structura sistemului}
%1 -- ilustrarea arhitecturii generale și detaliate a sistemului implementat, care să evidențieze modulele componente și relațiile dintre acestea

%arhitectura generala: 
%-vncamt.dll 
%-c extension
%-python script

%explicatie
%caracteristici vncamt.dll
În figura \ref{general} sunt prezentate componentele generale ale sistemului. După cum se vede, Active Management Technology expune o interfață. Pentru realizarea unei conexiuni Virtual Network Computing se folosește portul 5900. Este necesară o configurare inițială a sistemelor de test. Acest lucru necesită două etape de configurare. Chiar dacă calculatoarele folosesc tehnologia Intel vPro, funcționalitatea de AMT este disponibilă doar după ce este pornită și configurată. În plus trebuie pornit portul folosit de VNC și setată parola pentru Remote Framebuffer. Acești pași se vor prezenta în detaliu în manualul de utilizare.

\begin{figure}
    \centering
    \includegraphics[width=1\textwidth]{general}
    \caption{\textit{Componentele sistemului}}
    \label{general}
\end{figure}

Biblioteca vncamt.dll este folosită pentru a realiza comunicarea dintre calculatorul aflat la distanță și calculatorul client. Aceasta trebuie integrată în aplicație conform documentației oferite de RealVNC.
 


%-comunicare dintre ele
%-handshake
%-desen \ref{vncarhi}
Factori importanți care diferențiază VNC de la alte sisteme de afișare de la distanță sunt după cum urmează:

\begin{itemize}
  \item Nicio stare nu este stocată viewer. Acest lucru înseamnă că se permite reconectarea  la desktop de la un alt computer și se facilitează continuarea utilizării din starea în care a fost lasat. Chiar cursorul va fi în același loc. Cu un alt tip de server, dacă PC-ul se blochează sau este repornit, toate aplicațiile de la distanță vor muri. Cu VNC ele continuă să ruleze.
  \item Acesta este mic și simplu. Un viewer Win32, de exemplu, ocupă aproximativ 150K în dimensiuni și poate fi rulat direct de pe un floppy. Nu necesită nicio instalare.
  \item Este cu adevărat independent de platformă. Un desktop care rulează pe o mașină de Linux sau Solaris poate fi afișat pe orice PC. Sau orice alte arhitecturi. Simplitatea protocolului îl face ușor de portat pe platforme noi. Există vieweri Java, care pot rula în orice browser capabil Java.
  \item Este partajat. Un desktop poate fi afișat și utilizat de mai multi telespectatori simultan, permițând realizarea de aplicații în stil cooperative work.
  \item Este gratuit. Există multe versiuni disponibile pentru download sub termenii GNU Public License.
\end{itemize}

\begin{figure}
    \centering
    \includegraphics[width=0.7\textwidth]{rfbarh}
    \caption{\textit{Comnunicare între client și server}}
    \label{vncarhi}
\end{figure}

În acest sistem este interesantă comunicarea furnizată de protocol. Figura \ref{vncarhi} prezintă puțin mai detaliat cum se realizează transferul buffer-ului video de pe mașina remote. Este de notat faptul că între aceste componente există și alte mesaje transmise, de exemplu clientul trimite evenimente de intrare către server. Deoarece se folosește protocolul Remote FrameBuffer datele sunt trimise în mod normal pe o rețea nesecurizată. 

Mesajele care transmit date despre pixeli sunt trimise în general de fiecare dată când framebuffer-ul aflat la distanță a fost actualizat. În cazul în care clientul a pierdut o parte din buffer, acesta poate să trimită o înștiințare către server și să ceară să îi trimită acea porțiune. Serverul poate să trimită zona cerută sub orice fel de configurație de dreptunghiuri dorește. De asemenea, clientul are posibilitatea să trimită o cerere de update către server și în cazul în care nu a pierdut informațiile. Singura diferență față de cea anterioară este că de această dată serverul este înștiințat că nimic nu a fost pierdut. Astfel el nu o să trimită aceea zonă, ci o să aștepte până în momentul în care este modificată, și tocmai atunci o să trimită actualizarea către client.

Diagrama prezintă foarte bine componentele. Pe calculatorul aflat la distanță avem o aplicație care rulează. Aici se află serverul care așteaptă o conexiune. Pe baza informațiilor disponibile, se realizează un framebuffer virtual. Aici se află datele despre pixelii care se găsesc în acel moment pe consola video. Dacă viewer-ul s-a conectat cu succes, primește acces la acest buffer. Informația este mai întâi împachetată și apoi trimisă. Pe partea de client, viewer-ul oferă mai departe aceste date pentru a fi despachetate și interpretate. Se poate folosi orice fel de structură disponibilă pentru stocarea informațiilor despre pixeli, sau se poate crea o structură specială. Mai departe framebuffer-ul este disponibil și nu există nicio limitare pentru folosirea lui.






%arhitectura extensie: 
%-dbg 
%-metode specifice Python.h 
%-metode visibile in Python 
%-callback uri vncamt.dll

\section{Funcționare}
%2 -- stările prin care trece sistemul în decursul funcționării sale (diagrame de stare)

% conectare la real vnc, 

Aplicația inițial se află într-o stare de inactivitate. În această stare ea a inițializat deja biblioteca care o folosește și așteaptă comenzi. Pentru a reuși să stabilim o conexiune cu succes că să putem vedea frame buffer-ul sunt necesari mai multi pași. Mai întâi se creează un viewer VNC. Dacă operația a avut loc fără erori, atunci avem la dispoziție un obiect. Acesta poate fi distrus sau se poate da o comandă de pornire. Biblioteca când primește această instrucțiune cere mai întâi o autentificare. Dacă se introduc datele corect viewer-ul VNC devine activ. 

Această sesiune poate fi închisă explicit în momentul în care nu mai avem nevoie de ea. În cazul în care se uită închiderea explicită a viewer-ului, după o anumită perioadă de inactivitate viewer-ul este închis în mod automat. Acesta poate fi păstrat în această stare pentru a fi refolosit ulterior. În momentul în care se repornește, autentificarea nu mai este necesară.

\begin{figure}
    \centering
    \includegraphics[width=0.9\textwidth]{state}
    \caption{\textit{Diagrama de stare a proiectului}}
    \label{d:state}
\end{figure}



Dacă autentificarea nu s-a efectuat cu succes, suntem înștiințați. În acest moment, un viewer poate fi distrus, dar de cele mai multe ori este păstrat. Obiectul este folosit pentru a oferi mai multe reîncercări de autentificare. Este recomandată distrugerea viewer-ului atunci când știm cu siguranță că nu mai avem nevoie de el, de exemplu după ce conexiunea a fost oprită. În diagrama \ref{d:state} sunt prezentate stările de tranziție.



\section{Interacțiunea dintre module}
%3 --  modul de interacțiune dintre module și funcționalitatea acestora ilustrată prin diagrame de secvențe

% interfatarea cu metodele din extensia C

În această secțiune sunt prezentate două din operațiile de bază. Orice conexiune este constituită dintr-un client VNC care se conectează la un server VNC. Server-ul este aflat pe mașina remote și rulează acolo. Pe mașina client mai întâi trebuie creeat un viewer. Acest viewer este ulterior folosit, și distrus în momentul în care nu mai este folosit. În figura \ref{seq:create} se prezintă diagrama de secvență a acestui proces. Trebuie avut în vederere că script-ul handler.py și clasele VNCVIewerSDK, VNCViewer sunt într-un mediu Python, iar celelalte în mediul C. Deși pare o operațiune trivială la prima vedere, este necesară o bună comunicare pentru un management al erorilor eficient.

\begin{figure}
    \centering
    \includegraphics[width=1\textwidth]{create}
    \caption{\textit{Diagrama de secvență pentru realizarea unui Viewer}}
    \label{seq:create}
\end{figure}


Figura \ref{seq:start} prezintă diagrama de secvență pentru pornirea unui viewer VNC. Se pot observa operațiile necesare pentru funcția de start. Se transmite comanda de începere. Se interoghează o listă care conține toate viewer-ele și se returnează cel dorit. Cu ajutorul acestui obiect putem să continuam procesul prin intermediul bibliotecii. Callback-ul de autentificare din aplicația este apelat. Aici se furnizează o parolă, și dacă este cazul un user, pentru Remote Framebuffer. Atunci când procesul de autentificare se încheie cu succes se apelează un alt callback. De această dată este chemată funcția de inițializare. Această oferă informații esențiale despre datele care vor fi primite. Tot aici este inițializată structura care reține buffer-ul video.

\begin{figure}
    \centering
    \includegraphics[width=1\textwidth]{start}
    \caption{\textit{Diagrama de secvență pentru pornirea unui Viewer}}
    \label{seq:start}
\end{figure}

În general cele două operații sunt chemate consecutiv. Nu există un motiv pentru care să se creeze un viewer fără să fie pornit, și în mod sigur nu o să fie posibilă apelarea operației de pornire în cazul în care nu există un viewer care să fie pornit. Se poate reutiliza un obiect de tip viewer în cazul în care a fost oprit anterior, și acum este necesară pornirea acestuia.

%\section{Algoritmi?}
%%4 -- descrierea algoritmilor/metodelor pe care se bazează funcționarea sistemului dezvoltat 
%
%% algoritmul de desenat
%pixel buffer and raw encoding

\section{Organizare}
%5 -- descrierea organizării/structurii eventualelor baze de date folosite

% o arhitectura mai detaliata a:
%- py scripts: handler, viewer, pixelbuffer,
%- c extension: dbg.h,real vnc headers(*.h), vncmodule.c, vncmodule.h
%- helpers: setup.py, postbuild.bat

Figura \ref{exte} prezintă arhitectura proiectului de licență. După cum se bine poate observa, are trei mari componente. Header-ele librăriei \textit{vncamt.dll} comunică cu extensia scrisă în limbajul C. Acesta din urmă este practic un wrapper, care oferă mediului python acces la funcțiile librăriei. Mai există o a patra componentă care nu are direct legătură cu proiectul. Aceasta oferă ajutor în alte situații, cum ar fi o metodă automată de build.

\begin{figure}
    \centering
    \includegraphics[width=0.8\textwidth]{extensie}
    \caption{\textit{Arhitectura sistemului}}
    \label{exte}
\end{figure}

Pe partea de Python se disting 3 script-uri. Acestea nu sunt necesare, dar ușurează foarte mult integrarea. Fișierul \textbf{handler.py} oferă acces rapid la funcționalitățile pachetului. El reprezintă practic un exemplu care demonstrează cât de ușoară poate fi folosirea metodelor expuse. Fișierul \textbf{viewer.py} este cel care se ocupă de comunicarea cu extensia, este cel care abstractizează mare parte din proiect și îl face ușor de accesat, și nu în ultimul rând este cel care se ocupă de acțiunea de desenat. \textbf{pixelbuffer.py} expune o clasă special adaptată pentru necesitățile acestui proiect. Aceasta are capacitatea atât de a reține informații specifice, cât și de a decodifica inputul primit de viewer. Clasa \textit{Pixelbuffer} primește un stream de la viewer, îl decodifică și realizează o matrice ce poate fi ulterior interpretată de viewer, și desenată.

În general, un header notifică compilatorul de anumite lucruri, în cea mai mare parte de existența lor sau declarațiile, astfel încât compilatorul poate construi în mod corect o singură unitate de traducere. O bibliotecă este codul executabil real, care funcționează după cum se specifică în header. Acest lucru este legat de către linker pentru a oferi funcționalitatea reală. Deci pentru a utiliza tehnologia oferită de RealVNC trebuie să includem header-el librăriei \textit{vncamt}. Acestea includ structuri și antete de metode necesare integrării. Biblioteca \textit{vncamt} oferă macro-uri, definiții de tipuri de variabile, și funcții pentru sarcini cum ar fi manipularea de string-uri, prelucrarea intrărilor și ieșirilor, alocarea de memorie și alte câteva servicii.

Extensia este în mod sigur partea cea mai interesantă din acest proiect. Această componentă este podul dintre două lumi diferite. De o parte trebuie să aibă grijă de probleme specifice limbajului Python, cum ar fi valoarea reference conter-ului. De altă parte trebuie să țină cont de modul în care o aplicație în limbajul C necesită un management bun al erorilor. Exceptând librăria, o parte din codul extensiei este dedicat API-ului de C. Pentru crearea acestui modul implementarea C trebuie să o anumită structură. În primul rând proiectul trebuie sa includă fișierul \textit{Python.h}. Mai departe trebuie scrisă o metodă de inițializare a modului. La aceasta se adaugă funcțiile care vrem să le expunem. În plus este necesară crearea unui tabel care mapează numele funcțiilor expuse în Python cu numele funcțiilor C din interiorul extensiei.


\section{Global Interpreter Lock}

În CPython, Global Interpreter Lock sau GIL, este un mutex care previne mai multe thread-uri să execute bytecode Python în același timp. Aceast lacăt este necesar în principal din cauza faptului că managementul de memorie  CPython nu este thread-safe.
În implementarea acestui proiect s-au folosit metodele \textit{PyGILState\_Ensure()} pentru a prelua lacătul, și \textit{PyGILState\_Release()} pentru a-l elibera. \textit{PyGILState\_Ensure()} returnează un obiect de tipul \textit{PyGILState\_State}.
 
GIL urmărește câteva reguli simple:
\begin{itemize}
  \item execuția în paralel este interzisă
  \item există un \textit{global interpreter lock}
  \item GIL asigură că doar un singur fir se execută în
interpretor la orice moment dat
  \item simplifică multe detalii de nivel scăzut: management de memorie, trimitere de înștiințări extensiilor C, etc.
\end{itemize}

GIL din Python este destinat serializării accesului la componentele interne din thread-uri diferite. În cazul sistemelor cu mai multe procesoare, aceasta înseamnă că mai multe thread-uri nu pot face în mod eficient uz de mai multe nuclee. 

GIL Python este cu adevărat numai o problemă pentru CPython, implementarea de bază. \textit{Jython} și \textit{IronPython} nu au GIL. Ca dezvoltator Python, nu intri în contact cu GIL cu excepția cazului când trebuie scrisă o extensie C. Este nevoie să se elibereze GIL când extensiile împiedică operațiile I/O, astfel încât alte thread-uri de Python să obțină o șansă.

Multithreading-ul este posibil în Python, dar două thread-uri nu pot fi executate în același timp, la o granularitate mai mică decât o instrucțiune Piton. Thread-ul de execuție primește lacătul. Dacă scopul este să scriem cod care să profite de avantajele muti-core, atunci trebuie menționat că performanța nu se va îmbunătăți. O soluție de obicei constă în folosirea mai multor procese. Nu trebuie uitat să se elibereze lacătul.


\section{Justificări}
%6 -- justificarea alegerilor/deciziilor făcute și analiza critică a acestora (avantaje și dezavantaje), prin comparație cu alte alternative posibile

% pont: cauta prin comentariile de pe jira	

Una din primele soluții propuse pentru proiectul de licență a fost implementarea unei metode similare cu unul din proiectele Python deja existente. Design-ul poate fi urmărit cu ușurință folosind documentația de la protocolul Remote Framebuffer. Problema care a existat la aceste proiecte este faptul că ele au anumite dependințe. Lucrarea de față trebuie să fie scrisă în Python 3, iar resursele sunt scrise în Python 2. Există mici diferențe între versiuni care pot fi remediate, însă toate proiectele de VNC scrise în Python folosesc doua librării externe. Una din ele, și anume \textit{twisted}, este compatibilă momentan doar pentru prima versiune. În plus această abordare ar fi consumat mult timp pentru implementare. Din același motiv s-a renunțat și la alternativa unui modul de networking similar \textit{twisted}, cum ar fi \textit{gevent} sau \textit{asincyo}.

Prima încercare de scriere a unui wrapper a fost realizată cu \textit{ctypes}. Aceasta este o bibliotecă străină pentru Python. Oferă tipuri de date compatibile C și permite apelul funcțiilor unei biblioteci. Aceasta poate fi folosită pentru a scrie un wrapper doar în Python. S-a renunțat la această soluție atunci când a fost necesară scrierea unei definiții pentru o structură opacă utilizată de dll. \textit{ctypes} s-a dovedit ușor de înțeles și folosit. Oferă chiar posibilitatea descrierii structurilor din limbajul C. S-a reușit cu succes apelul unor funcții, însă s-a dovedit ineficient în transmiterea pointerilor. Alternativa aleasă a oferit o gamă mult mai largă de personalizare a opțiunilor. Astfel s-a început implementarea folosind Python C-API.

Mult timp s-a irosit din cauza incompatibilităților. Majoritatea versiunilor de Python mai mici de 3.3 au fost create folosind Microsoft Visual Studio 2008. De cele mai multe ori developerii care dezvoltă aplicații în Python sunt utilizatori de Linux. În mod normal acest lucru a dus la un deficit pentru acea comunitate care folosește Windows. Cu toate acestea aproape toate modulele importante se găsesc recompilate și pentru Windows. Un pas important a fost momentul în care s-a publicat Microsoft Visual Sudio 2015 și Python a devenit compilat cu acesta. Cu această ocazie s-a realizat migrarea proiectul. Până atunci a fost necesar folosirea unui compilator MingW special instalat și configurat.

Din lipsa surselor de informare, inițial a existat o perioadă de acomodare cu arhitectura unui astfel de modul. S-a creat scheletul proiectului și au fost implementate cu succes o serie de operațiuni de bază: apelarea unor metode Python, interpretarea de argumente, pasarea de informații, operațiuni cu pointeri. Implementarea callback-urilor a fost puțin mai dificilă. În momentul în care librăria \textit{vncamt} apelează un callback din extensia C, acesta apelează o funcție din Python cu exact aceleași argumente. Funcția din Python folosește aceste date și uneori poate chiar să returneze valori, care să fie trimise înapoi la callback din C. Abia acum acesta își încheie execuția.

%O altă problemă a fost în întâmpinată atunci când s-a realizat prima desenare a unui pătrat de 64 x 64 pixeli de pe un calculator remote. 
%
%...
%
%...






%\chapter{Implementation Details}
\chapter{Detalii de implementare}
\label{cap:implementare}


%Ponderea acestui capitol relativ la întreaga lucrare este de 20-30\%.
%
%Conține detalii de implementare: 
%\begin{itemize}
%  \item organizarea codului sursă, organizarea logică a codului (module, ierarhii de clase)
%  \item descrierea claselor, funcțiilor, API-urilor importante ale aplicației
%  \item descrierea la nivel de implementare a algoritmilor principali
%  \item descrierea părților mai dificile
%  \item alte detalii de implementare relevante, specifice fiecărei aplicații
%\end{itemize}
%
%Descrierea implementării trebuie să reflecte modul în care ea corespunde (se mapează) design-ului. 
%
%Nu se vor da detalii irelevante. Descrierea codului trebuie gândită ca un ghid de parcurgere a codului sursă de către cineva care vrea să continue proiectul vostru. 
%
%Exemplu de cod:
%\lstset{language=C,frame=single, showstringspaces=false}
%\begin{lstlisting}
%# include <stdio.h>
%  
%int main (int argc, char **argv)
%{
%  int i;
%    
%  for (i=0; i<argc; i++)
%    printf("argv[%d] = %s\n", i, argv[i]);
%    
%  return 0;
%}
%\end{lstlisting}

%%%%%%%%%%%%%%%%%%%%%%%%%%%%%%%%%%%%%%%%%%%%%%%%%%%%%%%%%%%%%%%%%%%%%%
 %20-30\% --> 11-16.5pg
%%%%%%%%%%%%%%%%%%%%%%%%%%%%%%%%%%%%%%%%%%%%%%%%%%%%%%%%%%%%%%%%%%%%%%
%în dezvoltarea aplicației s-au folosit diverse librării. 
%-amt  sdk 11.0.0.35
%-py3.5 
%-vs2015
%-wheel ver
%-numpy..ver
%-pygame ver

Inițial s-a început scrierea proiectului folosind Microsoft Visual Studio 2012 și Python 2.7 pe sistemul de operare Windows 8. S-a dovedit foarte greu de întreținut. Compilarea era destul de greoaie. Necesita MingW și în unele cazuri de utilizare a proiectului pe alte calculatoarea se dovedea de cele mai multe ori imposibil. Migrarea proiectului la Microsoft Visual Studio 2015 și Python 3.5 a fost una benefică. Din rândul dependințelor nu au apărut probleme. Modulele externe de Python folosite sunt: wheel, numpy versiunea 1.11.0b2 și pygame 1.9.2a0 pentru Python 3.5 arhitectură x64. Proiectul a fost realizat cu următoarele setări în Visual Studio:  

\begin{itemize}
  \item Warning Level: \textbf{Level 4}
  \item Treat Warnings As Errors: \textbf{Yes}
\end{itemize}

De cealaltă parte, în proiectul de C există doua mari include-uri. Unul se referă la arhitectura modului, și anume Pyton.h, iar celalalt este modalitatea de comunicare a informațiilor de volum mare, API-ul de C pentru Numpy. 

Probabil cea mai importantă componentă din acest proiect este reprezentată de SDK. În dezvoltarea acestui proiect s-a folosit Intel AMT versiunea 11.0.0.35. Acesta oferă o abundență de funcții. Protocolul Remote Frambuffer folosit pentru VNC are versiunea 3.8. El este cel care se ocupă de comunicarea dintre sisteme.
Deși RFB la început a fost un protocol relativ simplu, acesta a fost îmbunătățit cu caracteristici suplimentare, cum ar fi transferurile de fișiere, compresii avansate și tehnici de securitate. Pentru a menține compatibilitate între mai multi clienți și servere VNC cu implementări diferite, clienții și serverele negociază o conexiune utilizând cea mai buna versiune RFB, și opțiunile de compresie și de securitate cele mai potrivite pe care le pot suporta ambele.

\begin{figure}
    \centering
    \includegraphics[width=0.7\textwidth]{vsproj}
    \caption{\textit{Structura proiectelor}}
    \label{vsp}
\end{figure}

Pentru structurarea aplicației, în mediul de lucru Visual Sutdio 2015, am ales să separ rolurile și să le structurez în proiecte după funcționalitatea acestora. În figura \ref{vsp} este prezentată structura temei, așa cum apare ea în Microsoft Visual Studio. Pentru proiectul în Python s-a folosit o extensie externă. Python Tools pentru Visual Studio este o extensie complet gratuită, dezvoltă și întreținută de Microsoft cu contribuții din partea comunității. Pe Github este disponibil un repository pentru a vizualiza sau pentru a participa la dezvoltare.



\section{Structura modulului}

Este nevoie să include fișierul \textit{Python.h} în fișierul sursă C, care oferă acces la API Python intern utilizat pentru a face legătura dintre modulul și interpretor. Trebuie asigurat că \textit{Python.h} se include înainte de orice alt antet de care am putea avea nevoie. 

Fiecare metodă returnează un obiect Python. Nu există nici un astfel de lucru ca o funcție de tip void în Python, spre deosebire de C. În cazul funcțiilor care nu necesită să returneze o valoare, întoarce echivalentul din C  pentru valoarea \textit{None} din Python. Header-ele Python definesc un macro, Py\_RETURN\_NONE, care face acest lucru.

Maparea metodelor se face cu un array simplu de structuri PyMethodDef. Aici se găsesc toate metodele C. Acestea, de obicei, sunt denumite prin combinarea de nume de module și funcții, așa cum se arată aici:
\lstset{language=C,frame=single, showstringspaces=false}
\begin{lstlisting}
static PyObject*
modul_functie(PyObject *self, PyObject *args) 
{
	if (!PyArg_ParseTuple(args, "ii", &x, &y)) 
	{
      return NULL;
   	}
    /* ... */
    Py_BuildValue("i", z);
}
\end{lstlisting}

Pentru parsarea argumentelor în momentul în care o funcție se apelează din Python se poate utiliza funcția PyArg\_ParseTuple API pentru a extrage ponterul PyObject trimis funcției C. Primul argument al metodei PyArg\_ParseTuple este \textit{args}. Acesta este obiectul de parsat. Al doilea argument este un șir ce descrie formatul argumentelor ce trebuie așteptate. Fiecare argument este reprezentat de una sau mai multe caractere. Ele denumesc tipuri de date: integer, double, string, PyObject. Această funcție returnează 0 în caz de eroare, sau o valoare diferită de 0 pentru succes. Tupla este obiectul PyObject care a fost pasat în format drept al doilea argument. Py\_BuildValue primește un șir de formatare similar cu PyArg\_ParseTuple. În loc să paseze adresele către valorile, se transmit valorile reale. Variabila \textit{self} nu este unică limbajului Python. Această idee a fost împrumutată de la \textit{Modula-3}, după ce s-a dovedit utilă într-un use case. Nu există o declarare explicită a variabilei în Python și nici nu este un keyword Python dedicat.

\section{Contorul de referințe}

Toate obiectele Python, chiar și întregii, au un tip și un număr de referință. Tipul unui obiect determină ce fel de obiect este, de exemplu un număr întreg ,o listă sau o funcție definită de utilizator. Pentru fiecare dintre tipurile de date cunoscute există un macro ce verifică dacă un obiect este de acest tip.

Numărul de referință este important deoarece mărimea memoriei calculatoarelor este finită. Contează în cât de multe locuri există câte o referință la un obiect. Un astfel de loc ar putea fi un alt obiect, sau o variabilă globală C, sau o variabilă locală în unele funcții C. Atunci când numărul de referință a unui obiect devine zero, obiectul este dealocat. În cazul în care conține referințe la alte obiecte, numărul de referință este decrementat. Acele alte obiecte pot fi dealocate, la rândul lor, în cazul în care acest decrement face ca numărul lor de referință să devină zero, și așa mai departe.

Contorul de referință este întotdeauna manipulat în mod explicit. În mod normal se utilizează un macro \textit{Py\_INCREF()} pentru a incrementa numărul de referință al unui obiect cu unu, și \textit{Py\_DECREF()} pentru a decrementa cu o unitate. Macroul \textit{Py\_DECREF()} este mult mai complex decât cel de incrementare, deoarece trebuie să verifice dacă numărul de referință devine zero și apelează dealocatorul obiectului. Dealocatorul este un pointer al unei funcții din structura obiectului. Dealocatorul are grijă de decrementarea referințelor pentru celelalte obiecte din interiorul obiectului în cazul în care acesta este un tip de obiect compus, cum ar fi o listă. În plus se ocupă de executarea oricărei operații suplimentare de finalizare. Nu există nici o șansă ca numărul de referință să facă overflow. Se folosesc cel puțin la fel de multi biți pentru a salva reference counter-ul cât există locații de memorie distincte în memoria virtuală există.

Atunci când o funcție pasează o referință la un obiect, există două posibilități: funcția fură o referință la obiect, sau nu. Furt de referință înseamnă că atunci când pasezi o referință la o funcție, funcția presupune că deține acum această referință, și nici nu mai suntem responsabili de ea. Câteva funcții fură referințe. Cele două excepții notabile sunt \textit{PyList\_SetItem()} și \textit{PyTuple\_SetItem()}. De exemplu \textit{PyInt\_FromLong()} returnează o nouă referință, care poate fi imediat furată de \textit{PyTuple\_SetItem()}. Dacă se dorește să se continue utilizarea unui obiect, deși referința la acesta va fi furată, se folosește \textit{Py\_INCREF()} pentru a incrementa înainte de a apela funcția care fură referința.


Există o funcție generică, \textit{Py\_BuildValue()}, care poate crea obiecte mai comune folosind valorile din C. Folosește un string de formatare. De exemplu, cele de mai valori care au fost folosite pentru a crea un obiect Python, pentru a fi trimis în Python:
\lstset{language=C,frame=single, showstringspaces=false}
\begin{lstlisting}
/* ... */
 arglist = Py_BuildValue("(O)", context->self);
 result = PyObject_CallObject(py_CredentialsCallback, arglist);
/* ... */
}
\end{lstlisting}



\section{Structuri de date proprii}

Din cauza regulilor de design impuse de API-ul de C pentru Python și de librăria vncamt, acest folosește variabile globale. Majoritatea ar putea obiecta pentru că este bad practice. Aceste variabile sunt accesibile din tot programul. S-a optat pentru această opțiune deoarece este nevoie ca structura VNCViewerSDK ce conține toți pointerii către funcțiile sdk-ului după inițializare să fie accesibilă.

Fiecare viewer care este creeat are un set de callback-uri care le implementează. Aceste callback-uri au un context. În momentul în care sunt apelate, se transmite ca parametru un pointer către o structură specială. Această structură este înregistrată folosind metoda \textit{VNCViewerSetContext()} din SDK. Următoarea secvență de cod are rolul de a defini structura cu toate datele necesare.
\lstset{language=C,frame=single, showstringspaces=false}
\begin{lstlisting}
typedef struct _VNC_VIEWER_CONTEXT
{
    char		*ipAddress;                    
    PyObject		*self;                     
    VNCViewer		*pViewer;                 
    VNCViewerCallbacks	vncCallbacks;    
} VIEWER_CONTEXT, *PVIEWER_CONTEXT;
\end{lstlisting}

Se memorează pentru fiecare viewer o adresă de IP, care are avantajul ca e unică. De asemenea sunt înregistrate pointerul către viewer-ul de VNC și callback-urile acestuia. Obiectul Python din structură este de fapt pointer la obiectul VNCViewer corespunzător din Python. Este foarte util în momentul în care se apelează callback-urile. În Python metodele primesc acest pointer, și cu ajutorul unui mic artificiu știu direct să actualizeze instanța de PixelBuffer corespunzptoare.

Pentru a putea avea un management al acestor structuri a fost necesară crearea unor liste în care să fie salvate. S-a optat pentru o listă dublu înlănțuită deoarece oferă o flexibilitate sporită: 
\lstset{language=C,frame=single, showstringspaces=false}
\begin{lstlisting}
typedef struct _LIST_T
{
    struct _LIST_T	*next;
    struct _LIST_T	*prev;
    PVIEWER_CONTEXT	context;
}LIST_T, *PLIST_T;
\end{lstlisting}

Au fost implementate metodele necesare management-ului. Pentru a crea o listă, folosim metoda \textit{ListInsert()} deoarece inserează noile elemente în fața listei. Acest lucru se realizează deoarece acest ultim viewer creat are cele mai mai șanse să fie folosit următorul. În momentul în care avem nevoie de un viewer se poate apela \textit{ListGetNode()}, iar pentru șterge nodul din listă și pentru a elibera memoria se folosește \textit{ListDeleteNode()}.


\section{Integrarea bibliotecii vncamt}
%incarcarea librariei conform docs
%implementarea callback-urile. -respecta definitia din .h -este proprie
%inregistrarea lor
%partile comunica prin functii si callback uri

SDK-ul RealVNC Viewer permite crearea și manipularea de obiecte VNC. Un viewer VNC se poate conecta la un server VNC, și poate afișa conținutul ecranului serverului VNC pe propriul ecran, fie prin folosirea ferestrei desktop implicite, fie prin crearea unei ferestre care să afișeze notificările de update. 

%figura


În figura \ref{cbs} se prezintă interacțiunea dintre librăria vncamt și aplicație. Pentru a încărca librăria se folosește funcția \textit{LoadLibrary()}. Se localizează funcția de inițializare a bibliotecii, adică \textit{VNCViewerSDKInitialize()}, și se apelează \textit{GetProcAddress()}. Există o structură importantă. După inițializare structura \textit{VNCViewerSDK} este populată cu toate celelalte funcții ale librăriei. Acestea presupun manipularea unu viewer. Acest VNC viewer este o structură opacă. Pentru crearea unui astfel de obiect trebuie apelată metoda \textit{VNCViewerCreate()}. Această funcție necesită transmiterea structurii \textit{VNCViewerCallbacks} pentru a furniza toate adresele funcțiilor ce trebuie înregistrate. Nu toate sunt obligatorii, dar există trei funcții care trebuie tot timpul implementate. \textit{VNCViewerServerInitCallback} notifică clientul că a fost realizată o conexiune cu succes. \textit{VNCViewerStatusCallback} oferă notificări în momentul în care o autentificare a eșuat sau când o conexiune a fost pierdută. \textit{VNCViewerCredentialsCallback} este folosit de user pentru a introduce o parolă în cazul în care serverul cere acest lucru. Mai departe se apelează metoda \textit{VNCViewerStart()} pentru a porni VNC viewer-ul. Se realizează conexiunea dintre clientul VNC și serverul VNC. Dacă a fost cu succes, utilizatorul poate să vadă desktop-ul remote la el pe calculator. Trebuie reținut faptul că există un timeout. În cazul în care nu se apelează în mod explicit \textit{VNCViewerStop()}, conexiunea se închid din motive de inactivitate. În ambele cazuri \textit{VNCViewerStatusCallback()} oferă un mesaj de informare despre oprirea viewer-ului. Acest callback este foarte util atunci când avem nevoie să facem un management al resurselor. De exemplu se poate apela funcția \textit{VNCViewerDestroy()} pentru a șterge VNC viewer-ul, sau se poate reapela callback-ul de autentificare pentru a păstra obiectul și a îl refolosi. Așa cum bine se poate observa comunicarea este în ambele sensuri într-o manieră organizată.

\begin{figure}
    \centering
    \includegraphics[width=0.55\textwidth]{callback}
    \caption{\textit{Integrarea librăriei}}
    \label{cbs}
\end{figure}



\section{Structuri de date importate}

NumPy oferă un API de C pentru a permite utilizatorilor să extindă sistemul și să obțină accesul la obiect de tip matrice pentru a fi utilizate în alte rutine. 
Câteva tipuri noi sunt definite în C. Cele mai multe dintre acestea sunt accesibile din Python, dar câteva nu sunt expuse datorită utilizării lor limitate. Fiecare nou tip Python are un PyObject * asociat cu o structură internă care include un pointer la o tabelă de metode, ce definește modul în care noul obiect se comportă în Python. Când se primește un obiect Python în cod C, se obține întotdeauna un pointer la o structură PyObject.

Există două mari tipuri noi: ndarray( \textit{PyArray\_Type} ) și ufunc( \textit{PyUFunc\_Type} ). Cel folosit în proiect este primul. PyArrayObject este o structura C oferită de API ce poate ține toate datele despre pixeli și poate fi interpretată și în Python. Aceasta conține toate informațiile necesare pentru o matrice. Toate instanțele unui ndarray, și subclasele sale, au această structură. Pentru compatibilitate viitoare, aceste elemente ale structurii ar trebui să fie accesate în mod normal folosind macro-urile oferite. Pentru un nume mai scurt se poate folosi NPY\_AO care este definit echivalent cu PyArrayObject. Arată astfel:

\lstset{language=C,frame=single, showstringspaces=false}
\begin{lstlisting}
typedef struct PyArrayObject {
    PyObject_HEAD
    char *data;
    int nd;
    npy_intp *dimensions;
    npy_intp *strides;
    PyObject *base;
    PyArray_Descr *descr;
    int flags;
    PyObject *weakreflist;
} PyArrayObject;
\end{lstlisting}

Rutina \textit{PyArray\_SimpleNewFromData()} face într-un mod simplu conversia unei zone de memorie alocate într-un obiect ndarray pentru a fi folosit mai tarziu. Primele trei argumente descriu dimensiunea și tipul valorilor din matrice, iar ultimul argument este un pointer la un bloc de memorie contiguă ce ndarray-ul ar trebui să-l utilizeze ca buffer pentru a fi interpretate. Se returnează o referință la un ndarray, dar trebuie avut în vedere că ndarray nu își deține propriile datele. Atunci când este dealocat, pointer-ul nu este eliberat. Atâta timp cât ndarray este folosit, trebuie avut grijă ca memoria să nu fie eliberată. Pentru acest lucru este necesar să se incrementeze reference counter-ul.

Modulul Python de Numpy oferă și acesta funcționalități importante. ndarrays pot fi indexate folosind sintaxa standard Python \textit{x[i]}, unde \textit{x} este o matrice și \textit{i} indexul. Există trei tipuri de indexare disponibile:
\begin{itemize}
  \item acces normal, explicit se alege numărul indexului.
  \item basic slicing, are sintaxa de bază de forma \textit{i:j:k}, unde i reprezintă indexul de început, j desemnează ultimul index, iar k este dimensiunea incrementului. Acesta trebuie să fie diferit de 0. Pot exista și valori negative.
  \item indexarea avansată este declanșată atunci când obiectul de selecție, \textit{i}, este un obiect secvență non-tuplă, un ndarray de tipul de date întreg sau boolean, sau o tuplă cu cel puțin un obiect de secvență. Există două tipuri de indexare avansate: întreg și boolean. Indexare matricei cu întregi permite selectarea elementelor arbitrare din matrice pe baza indicelui lor N. Fiecare matrice reprezintă un număr de indici în acea dimensiune. Indexare avansat apare atunci când \textit{i} este un obiect array de tipul boolean, poate fi returnat folosind operatori de comparare.
\end{itemize} 

Pachetul Numpy oferă o multitudine de funcții pentru operații cu array-uri. Cele mai importante sunt operațiile logice și cele aritmetice. Acestea sunt folosite în implementarea clasei \textit{PixelBuffer} pentru a realiza conversia valorilor de RGB.




\section{Buffer-ul video}

Protocolul de actualizare este bazat pe cerere client. Aceasta este o actualizare trimisă numai
de la server la client ca răspuns la o cerere explicită din partea clientului. Asta dă protocolului o calitate adaptivă. Schimbările unei zone de framebuffer tind să se întâmple consecutiv. Cu un client sau
de o rețea lentă, stările tranzitorii ale framebuffer-ului pot fi pierdute, rezultând în trafic mai mic de rețea și mai puține operații de desenare pentru client.

\begin{figure}
    \centering
    \includegraphics[width=0.7\textwidth]{pythonside}
    \caption{\textit{Clasele din Python}}
    \label{pyside}
\end{figure}

Pixelbuffer este o clasă special realizată în Python. Ea a fost gândită în așa fel încât să faciliteze comunicarea cu ajutorul protocolului Remote Framebuffer. Obiectele de tip PixelBuffer sunt utilizate în fiecare instanță de tip VNCViewer. În figura \ref{pyside} sunt detaliate aceste clase. Clasa VNCViewerSDK
are în primul rând rolul de a apela metoda din extensie de inițializare a librăriei vncamt, făcând astfel disponibile funcțiile, și pe urmă un rol de management al viewerelor VNC. Acestea sunt abstractizare în clasa VNCViewer. Acest obiect conține metode care apelează biblioteca prin intermediul extensiei și descrie comportamentul callback-urilor din Python. Mai are atribute ce conțin informații despre el, dar cel mai important dintre ele este buffer-ul video. Instanța de PixelBuffer este inițializată în callback-ul de inițializare al SDK-ului. Aici sunt expuse opțiunile cu care a fost realizată conexiunea dintre clientul și serverul de VNC. Datele stocate sunt esențiale la interpretarea pachetelor primite de la server: lățimea și înălțimea ecranului în pixeli, valorile de maxim și cele de shift pentru roșu, verde, albastru, adâncimea, biți pe pixel.

Atunci când are loc un update de buffer video, unul din cele trei callback-uri este apelat: ImageRectangle, CopyRectangle, FillRectangle. Au o funcționare similară, și anume toate notifică viewer-ul interesat despre poziția unde are loc modificarea și datele ce trebuie interpretare. În consecință clasa PixelBuffer are implementate trei metode pentru fiecare din acestea situații:

\begin{itemize}
  \item Funcția \textit{update\_rectangle()} primește două seturi de coordonate. Colțul din stânga sus și cel din dreapta jos al dreptunghiului ce trebuie actualizat. Folosind acestea PixelBuffer știe exact care este submatricea ce trebuie modificată. Informațiile despre pixelii ce trebuie modificați sunt deja transmiși din C sub formă de matrice, ceea ce ușurează o parte a funcționalității.
  \item Spre deosebire de funcția anterioară, \textit{fill\_rectangle} primește o submatrice a cărui elemente reprezintă aceleași valori ale unui pixel.
  \item \textit{copy\_rectangle()} este notificat despre coordonatele sursă și cele de destinație al unui dreptunghi. De exemplu, în cazul care o fereastră este mutată pe ecran.
  \item Există o metodă bonus. Aceasta se numește \textit{resize\_screen()} și este apelată în momentul în care rezoluția monitorului remote se schimbă. Doar în acest caz buffer-ul este aruncat și înlocuit de unul care are noile dimensiuni de înălțime și lățime corespunzătoare
\end{itemize}

Aceste metode oferă o optimizare a operațiilor deoarece sunt actualizate doar zonele de interes al buffer-ului, nu se reface toată matricea de fiecare dată când există o modificare. Informațiile despre pixeli sunt stocate sub forma \ref{pixdata}.
 
\begin{figure}
    \centering
    \includegraphics[width=0.65\textwidth]{pixeldata}
    \caption{\textit{Reprezentarea internă a datelor despre pixeli}}
    \label{pixdata}
\end{figure}

Elementele matricei sunt pixelii de pe ecran. Formatul pixelilor este de fapt o reprezentare individuala a valorilor culorilor pixelului. În Python aceste valori sunt reținute sub forma unor tuple. Pentru a ajunge în aceasta stare ușor de interpretat de numpy pentru a fi mai apoi desenată imaginea. Cele mai comune formate sunt pe 16 biți. Pentru a extrage aceste valori din stream și pentru a le transforma este nevoie de un proces de conversie care PixelBuffer îl folosește.

Mai jos se arată modul de calcul al valorilor culorilor pixelilor:
\lstset{language=C,frame=single, showstringspaces=false}
\begin{lstlisting}
red = ((pixelData >> redShift) & redMax) * 255 / redMax
red = red << 16
  
green = ((pixelData >> greenShift) & greenMax) * 255 / greenMax
green = green << 8
  
blue = ((pixelData >> blueShift) & blueMax) * 255 / blueMax
\end{lstlisting}

Există mai multe tipuri de codificări și implicit mai multe metode de conversie. La momentul actual proiectul implementează doar această variantă. În cazul în care se decide că e necesară integrarea unei noi codificări, acest lucru se face destul de simplu urmărind arhitectura actuală și regulile de conversie alese.





\section{Interfață utilizator}

Pygame este un wrapper pentru librăria SDL, Simple DirectMedia Layer, dar mai conține câteva librării unice pentru programarea de jocuri. Pachetul are o structură proprie ce trebuie implementată. În primul rând este importat modulul în Python. 
Pentru a inițializa Pygame se apelează metoda \textit{init()}. Trebuie setată o variabilă cu dimensiunile ferestrei, \textit{display.set\_mode()}. Mai departe se rulează programul în buclă până când primește un eveniment de terminare. La fiecare iterație sunt interceptate evenimentele pentru a fi tratate în consecință, iar variabila setată anterior este actualizată cu noile culori sau primitive grafice. Pentru a face vizibile aceste actualizări se apelează funcția \textit{display.flip()}. Mai departe se exemplifică arhitectura de bază:
\lstset{language=Python,frame=single, showstringspaces=false}
\begin{lstlisting}
import pygame

pygame.init()
screen = pygame.display.set_mode((800, 400))

runnig = True
while running:
  for event in pygame.event.get():
    if event.type = pygame.QUIT:
      running = False
    # ...
  screen.flip((255,0,0))
  pygame.display.flip()      
\end{lstlisting}


Pachetul \textit{surfarray} din Pygame oferă funcții pentru a converti datele pixelilor din suprafețele pygame în matrice. Acest modul va fi funcțional doar atunci când pygame poate utiliza externe pachetele NumPy sau Numeric. Fiecare pixel este stocat ca o singură valoare întreagă pentru a reprezenta culorile roșu, verde și albastru. Pixeli cu adâncime mai mare pot utiliza un proces de împachetare pentru a plasa trei sau patru valori într-un singur număr. Matricele sunt indexate mai întâi de axa X, urmată de axa Y.

Inițial se foloseau mai multe metode din acest modul. Dintr-un anume motiv, după ce se creează un surface din matricea existentă, acesta trebuie răsturnat și rotit la 90 de grade, înainte să fie actualizat surface-ul original. Pentru a evita două transformi consecutive ale surface-ului, sau ale matricii, s-a folosit metoda \textit{blit\_array()}. Aceasta primește ca parametru matricea, și actualizează direct surface-ul, fără a fi necesare transformările.









%\section{lamda functions}

%\section{Extindere}






%\chapter{Tests and Results}
\chapter{Teste și rezultate experimentale}
\label{cap:rezultate}

%Ponderea acestui capitol relativ la întreaga lucrare este de 5-10\%.
%
%Aici sunt prezentate metodele de validare a soluțiilor/sistemului descris în capitolele anterioare, scenariile de testare a corectitudinii funcționale, a utilizabilității, performanței etc.   
%
%Rezultatele testelor experimentale necesită, în general interpretări (dacă rezultatele obținute corespund așteptărilor, intuițiilor cititorului, de ce apar variații/excepții etc.) și comparații cu rezultatele altor metode similare. 
%
%Sistemele de testare și testele propriu-zise trebuie descrise detaliat astfel încât să poată fi reproduse și de alții care poate vor să-și compare soluțiile lor cu a voastră (eventual, codul testelor poate fi pus în anexe). Dacă se poate alegeți pentru evaluarea sistemului vostru benchmark-uri (pachete de testare) dedicate, astfel încât comparația cu alte sisteme să poată fi făcută mai ușor. În plus, astfel de teste sunt mult mai complete și mai realiste decât cele dezvoltate de voi. Oricum, încercați ca testele efectuate să nu fie triviale, ci să acopere scenarii cât mai reale, mai complexe și mai relevante ale funcționării sistemului vostru. 

Testarea este un proces esențial în dezvoltarea de software. Scopul procesului de testare îl reprezintă izolarea și identificarea erorilor. Nu se poate garanta faptul că în urmă oricărui fel de examinări produsul o să fie lipsit de orice defect, dar cu siguranță se elimină o parte din greșelile sau neclaritățile cele mai evidente. Aceasta se datorează imposibilității de a utiliza produsul în toate combinațiile posibile, dar de multe ori este o cauză a încercărilor frecvente de automatizare de teste.

În dezvoltarea acestui proiect s-au aplicat diverse metode. În primul, și probabil cel mai important, vorbim despre testarea manuală și exploratory testing. Acest tip de testare presupune interacțiunea cu produsul într-un mod care exercită imaginația. Necesită și o bună cunoaștere a soft-ului. Prin acest tip de testare nu doar verificăm funcționalitatea, ci practic simulam o interacțiune a unui utilizator normal care are ocazia să pună sub stres diverse elemente ale proiectului. Cele mai interesante bug-uri apar de cele mai multe ori în urma acestui tip de testare.

De asemenea există și un tip de teste ce se numesc unit test. Acest tip de testare se referă la scrierea de test case-uri care verifică corectitudinea codului. O astfel de testare este o testare white box. Testerul are acces la codul sursă, și se folosește de diverse framework-uri pentru a apela funcțiile software-ului cu diverși parametrii de intrare, chiar și greșiți, pentru a compara ulterior rezultatul returnat cu rezultatul așteptat. Este o metodă bună pentru a acoperi cât mai mult cod și este foarte util din punct de vedere al vitezei. În scrierea acestui proiect nu am folosit foarte mult această metodă deoarece majoritatea funcțiilor sunt callback-uri apelate de librărie, care este opacă, iar cele scrise sunt prea triviale pentru a fi menționate.

În continuare se vor prezenta două tipuri de testare aleasă pentru acest proiect: testarea funcțională și testarea de performanță. Am ales acest tip de teste deoarece reprezintă un interes mai mare.

%\section{Functional Tests}
\section{Teste de funcționalitate}
%- c testam listele
%- python testam callback-uri
%	sunt apelate callback uri de init la init
%	sunt apelate callback status la close cu status de eroare
Testarea funcțională este un tip de testare black-box, în care testele sunt realizate pe baza specificațiilor. Nu trebuie confundat faptul ca s-au folosit cunoștințele despre clasele din Python cu testare white-box. Funcționalitatea care dorim să o verificăm este cea a modului scris în limbajul C și nu a funcțiilor ajutătoare utilizării modului scrise în limbajul Python.

Testele au fost scrise integral in Python. Pentru a ușura scrierea testelor s-a folosit un modul numit unittest. Acesta este un framework focusat pe unit testing, insipat de JUnit, și are cam aceleași caracteristici care le au și framework-urile cunoscute din alte limbaje. Există și alte alternative mai puternice și mai versatile, cum ar fi Letuce sau Behave, dar am ales unittest deoarece are uneltele necesare de care am avut nevoie în scrierea acestor teste și nu necesită instalări adiționale deoarece vine impreună cu Python. 

Scriptul de testare este unul extrem de simplu. În primă fază importa evident modulul unittest, declară clasele de test cu scenariile dorite, și apelează unittest.main(). Partea interesantă se regăsește în clasele de test. Aici am descris scenarii diverse. S-a oferit o lista de funcții de callback-uri custom, și s-a verificat faptul că acestea sunt apelate. În cazul în care callback-ul de credențiale oferă o parolă greșită, verificăm că nu s-a apelat callback-ul de inițializare și callback-ul de status a returnat un cod de eroare corespunzător cu scenariul. În cazul în care callback-ul de credențiale oferă o parolă corecta, verificăm că s-au apelat callback-urile specifice, iar în urma închiderii conexiunii, callback-ul de status a returnat un cod de eroare "Connection Closed".

%\section{Performance Tests}
\section{Teste de performanță}
%- wireshark stuff
%	standby
%	4k

Performance testing este un termen generic care se poate referi la mai multe tipuri de testare de performanță, fiecare adresându-se unui anumit tip de problemă. Sunt destinate pentru a determina capacitatea de reacție, debitul ,fiabilitatea și/sau scalabilitatea unui sistem în cadrul unui anumit volum de muncă. Acest tip de testare, indiferent cărui problemă se adresează, este important deoarece oferă o metrică asupra beneficiilor aduse sau nu în comparație cu produsele concurente.

în primă fază această etapă a reprezentat o provocare deoarece nu pare să existe o metodă evidentă de a compara proiectul cu alte produse similare. Lucrarea \cite{perf} a oferit o ideea cum s-ar putea realiza această comparație, analizând traficul. Rezultatele s-au dovedit a fi foarte interesante. Pentru a identifica căile de comunicație s-a folosit Process Monitor, un tool de la Microsoft care oferă informații despre ip-urile sau porturile folosite de diverse aplicații. Pentru restul procesului de analiză s-a folosit Wireshark, un software dedicat monitorizării protocoalelor network. Folosind informațiile oferite de Process Monitor am creat filtre pentru a izola pachetele în Wireshark. S-au folosit următoarele calculatoare pentru aceste teste:

Specificațiile calculatorului client:
\begin{itemize}
  \item DELL Optiplex 9020
  \item Windows 8.1 Enterprise x64
  \item Intel(R) Core(TM) i7-4790 CPU @ 3.60 GHz
  \item memory 8.00 GB
\end{itemize}

Specificațiile primului calculator client:
\begin{itemize}
  \item HP EliteDesk 800
  \item Windows 8.1 Pro x64
  \item Intel(R) Core(TM) i7-4790 CPU @ 3.60 GHz
  \item memory 4.00 GB
\end{itemize}

Specificațiile celui de al doilea calculator client:
\begin{itemize}
  \item HP EliteDesk 800
  \item Windows 8.1 Pro x64
  \item Intel(R) Core(TM) i5-4590 CPU @ 3.30 GHz
  \item memory 4.00 GB
\end{itemize}

%%%%%%%%%%%%%%%% TABEL 1
\begin{table}[t]
\centering
\begin{tabular}{|c|c|c|}
\hline
                                &  wrapper 		 & RealVNC \\ \hline
\textbf{Average pps}            & 17.8           & 6.2     \\ \hline
\textbf{Average packet size, B} & 726.5          & 213.5   \\ \hline
\textbf{Average bytes/s}        & 12 k           & 1331    \\ \hline
\textbf{Average bits/s}         & 103 k          & 10 k    \\ \hline
\end{tabular}
\caption{Idle: wrapper vs RealVNC}
\label{table:s1}
\end{table}
%%%%%%%%%%%%%% GRAFIC 1
\begin{figure}
    \centering
    \includegraphics[width=0.75\textwidth]{s1}
    \caption{Idle: wrapper(negru) vs RealVNC(roșu)}
    \label{fig:s1}
\end{figure}
%%%%%%%%%%%%%%% TABEL 2
\begin{table}[t]
\centering
\begin{tabular}{|c|c|c|}
\hline
                                & wrapper & RDC 	\\ \hline
\textbf{Average pps}            & 7.6	  & 3.5     \\ \hline
\textbf{Average packet size, B} & 202.5   & 158.5	\\ \hline
\textbf{Average bytes/s}        & 1549    & 552     \\ \hline
\textbf{Average bits/s}         & 12 k    & 4420	\\ \hline
\end{tabular}
\caption{Idle: wrapper vs Remote Desktop Connection}
\label{table:s2}
\end{table}
%%%%%%%%%%%%%% GRAFIC 2
\begin{figure}
    \centering
    \includegraphics[width=0.75\textwidth]{s2}
    \caption{Idle: wrapper(negru) vs RDC(albastru)}
    \label{fig:s2}
\end{figure}

Primul scenariu de test a presupus conexiunea simultană la cele două calculatoare de test după ce au bootat, în timp ce sunt inactive. La un calculator conexiunea s-a realizat cu proiectul personal, iar la celălalt cu software-ul de la RealVNC. După cum reiese din tabel tabelul \ref{table:s1} RealVNC se descurcă mai bine. În prima jumătate a testului ferestrele au fost minimizate, iar în a doua au fost maximizate. S-a descoperit că RealVNC cel mai probabil are funcții care reduc traficul atunci când nu este necesar, lucru observabil în graficul \ref{fig:s1}. 

Al doilea scenariu a reprodus scenariul anterior, de data asta comparând proiectul cu Remote Desktop Connection. Spre deosebire de RealVnc, Remote Desktop Connection deși se comportă mult mai bine, primește pachete în mod constant. Rezultatele \ref{table:s2} sunt vizibile foarte clar și în graficul \ref{fig:s2}.

%%%%%%% TABEL 3
\begin{table}[t]
\centering
\begin{tabular}{|c|c|c|}
\hline
                                & wrapper   & RealVNC   \\ \hline
\textbf{Average pps}            & 4536.3    & 1503.6    \\ \hline
\textbf{Average packet size, B} & 828.5	    & 775.5     \\ \hline
\textbf{Average bytes/s}        & 3757 k    & 1166 k	\\ \hline
\textbf{Average bits/s}         & 30 M      & 9328 k  	\\ \hline
\end{tabular}
\caption{4k video stream: wrapper vs RealVNC}
\label{table:s3}
\end{table}
%%%%%%%%%%% GRAFIC 3
\begin{figure}
    \centering
    \includegraphics[width=0.75\textwidth]{s3}
    \caption{4k video stream: wrapper(negru) vs RealVNC(roșu)}
    \label{fig:s3}
\end{figure}
%%%%%%%%% TABEL 4
\begin{table}[t]
\centering
\begin{tabular}{|c|c|c|}
\hline
                                & wrapper & RDC    \\ \hline
\textbf{Average pps}            & 1481.6  & 973.6  \\ \hline
\textbf{Average packet size, B} & 780.5   & 735.5  \\ \hline
\textbf{Average bytes/s}        & 1156 k  & 715 k  \\ \hline
\textbf{Average bits/s}         & 9253 k  & 5727 k \\ \hline
\end{tabular}
\caption{4k video stream: wrapper vs Remote Desktop Connection}
\label{table:s4}
\end{table}
%%%%%%%%%%% GRAFIC 4
\begin{figure}
    \centering
    \includegraphics[width=0.75\textwidth]{s4}
    \caption{4k video stream: wrapper(negru) vs RealVNC(albastru)}
    \label{fig:s4}
\end{figure}
Mai departe s-au ales să se compare simultan aplicațiile în momentul în care sunt solicitate. Scenariul presupune rularea fullscreen a unui video 4k de pe YouTube folosind browser-ul Mozilla Firefox pentru o durată de un minut. Mai întâi s-a comparat proiectul cu RealVNC. Și de această dată, conform rezultatelor vizibile atât în tabelul \ref{table:s3} cât și în graficul\ref{fig:s3}, RealVNC s-a descurcat mai bine. La fel de bine s-a comportat și Remote Desktop Connection, dupa cum bine se poate observa în graficul\ref{fig:s4} și tabelul\ref{table:s4}.

%%%%%% concluzii
\begin{figure}
    \centering
    \includegraphics[width=0.75\textwidth]{scache}
    \caption{Without caching: wrapper vs RealVNC}
    \label{fig:cache}
\end{figure}

Graficul \ref{fig:cache} reprezintă unul din primele teste realizate între proiect și RealVNC. De această dată valorile sunt foarte apropiate. Cel mai probabil RealVNC are un sistem de caching care ajută într-un mod semnificativ. Am ales se prezint rezultatele anterioare deoarece, pe termen lung, acestea sunt mai concludente. Din punct de vedere al unui utilizator normal, atât proiectul de licență cât și RealVNC se comportă la fel, în timp ce Remote Desktop Connection reușește chiar să ofere o experientă mult mai fluidă.

%User Manual
%chapter{User Manual}
\chapter{Manual utilizator}

\label{cap:user-manual}


%Dacă dezvoltarea aplicației s-a bazat sau a presupus instalarea și configurarea unei infrastructuri (complexe), descrieți detaliat pașii pe care i-ați urmat (referințele utilizate) și mai ales abaterile voite sau necesare de la documenațiile referite. Încercați ca cineva care vă continuă tema să nu mai fie nevoit să mai piardă timp inutil cu pregătirea mediului de lucru și să poată trece cât mai repede la abordarea temei proptriu-zise a proiectului. 
%
%Indincați, de asemenea, explicit versiunile aplicațiilor, bibliotecilor folosite și salvați o copie a acestora pe CD-ul atașat lucrării. E posibil ca aplicația voastră să nu mai funcționeze la fel pe alte versiuni și e bine de știut acest lucru și,  în același timp, e bine ca mediul descris de voi să poată fi reprodus ulterior. 
%
%Se întinde pe aproximativ 2-3 pagini. 


În acest capitol se descriu pașii de instalare și rulare a aplicației. Așa cum am menționat și în capitolele anterioare, proiectul a fost la un anumit punct migrat. Instrucțiunile sunt actualizare din acest punct de vedere. De asemenea sunt incluse versiunile actuale ale tuturor pachetelor externe folosite în dezvoltarea acestui proiect.

\section{Instalare}

\subsection{Configurarea mașinilor}

Pentru a folosi acest modul este nevoie în primul rând să avem tehnologia AMT disponibilă. Acest lucru necesită un pas de instalare, și unul de configurare. Instalarea se efectuează o singură dată. Presupune setarea sistemului și a conectivității network. După aceea Intel AMT poate fi accesat prin rețea de aplicații.

Acesta este un proces manual care se face prin \textbf{MEBx}. Pentru a accesa acest meniu se restartează calculatorul și se apasă \textbf{Ctrl + P}.

1. Se selecteaza \textbf{MEBx Login}, ca în figura \ref{pas1}.  
 
\begin{figure}
    \centering
    \includegraphics[width=0.65\textwidth]{pasul1}
    \caption{\textit{MEBx Login}}
    \label{pas1}
\end{figure}

2. La acest pas se setează o parolă pentru MEBx. Parola trebuie sa aibă între 8 și 32 de caractere, și trebuie să conțină litere mari, litere mici, numere și caractere speciale. Parola default este \textit{admin}. După schimbarea parolei trece sistemul de la modul Factory la modul In-Setup.

3. Din meniul principal se selectează \textbf{Intel AMT Configuration} ca în figura \ref{pas2}.

\begin{figure}
    \centering
    \includegraphics[width=0.65\textwidth]{pasul2}
    \caption{\textit{Intel AMT Configuration}}
    \label{pas2}
\end{figure}

4. Din aceste meniu, figura \ref{pas3}, se activează opțiunea \textbf{Manageability Feature Selection}.

\begin{figure}
    \centering
    \includegraphics[width=0.65\textwidth]{pasul3}
    \caption{\textit{\textbf{Reprezentarea internă a datelor despre pixeli}}}
    \label{pas3}
\end{figure}

5. Se intră în meniul \textbf{SOL / IDER / KVM} și se pornesc și acestea. După care ne se revine în meniul anterior.

6. La fel ca la pasul precedent, se setează în \textbf{User Consent} opțiunea \textbf{User Opt-In} cu valoarea \textbf{KVM}.

7. \textbf{Password Policy} trebuie să rămână pe valoarea \textbf{Anytime}.

8. Se selectează \textbf{Network Setup}

9. Se selectează \textbf{TCP / IP Settings}, din meniul \ref{pas4}

\begin{figure}
    \centering
    \includegraphics[width=0.65\textwidth]{pasul4}
    \caption{\textit{TCP / IP Settings}}
    \label{pas4}
\end{figure}

10. Mai departe se intră în meniul \textbf{Wired LAN IPV4 Configuration}

\begin{figure}
    \centering
    \includegraphics[width=0.65\textwidth]{pasul5}
    \caption{\textit{Wired LAN IPV4 Configuration}}
    \label{pas5}
\end{figure}

11. În acest meniu \ref{pas5} trebuie făcute mai multe setări. \textbf{DHCP Mode} trebuie să fie dezactivat. Mai departe în trece adresa IP statică pentru AMT în câmpul \textbf{IPV4 Address}. Trebuie completate și restul câmpurilor. Trebuie verificat mai întâi ca IP-ul folosit nu există în rețeaua în care se află sistemul.


Dupa ce Intel AMT a fost configurat putem să trecem mai departe. Pentru a seta o parola de Remote Framebuffer am folosit o aplicație din AMT SDK. Aceasta se numește KVMControlApplication.

\begin{enumerate}
  \item Se pornește aplicația. 
  \item Se selectează \textbf{Edit Machine Settins}
  \item În căsuța \textbf{Hostname / IP} se introduce adresa IP setată anterior pentru AMT
  \item Se apasă butonul \textbf{Machine Settings}
  \item În această fereastră se dă \textbf{Enable All Ports} și se setează parola pentru Remote Framebuffer. Este important de reținut că această parolă are o restricție în plus față de parola MEBx, și anume, trebuie să aibă exact 8 caractere.
\end{enumerate}

\subsection{Instalarea modului}

În cazul în care se dorește recompilarea sau modificarea proiectului, acesta a fost implementat folosind Microsoft Visual Studio 2015. Pentru a folosi acest modul, înainte de toate, trebuie instalat Python. Se recomandă versiunea 3.5 pentru arhitectura x64 deoarece s-a dovedit a fi compatibilă cu Visual Studio 2015. Proiectul folosește și alte module externe: wheel, numpy versiunea 1.11.0b2 și pygame 1.9.2a0. Pentru a le instala se poate folosi comanda \textit{pip install <pachet>}. După ce s-au rezolvat aceste dependințe se poate instala și modulul nostru.

\section{Utilizare}

Pentru a folosi modulul într-un script Python, la fel ca orice alt modul, acesta trebuie în primul rând importat într-un proiect. Înainte trebuie să ne asigurăm ca biblioteca vncamt.dll se află lângă script. Se recomandă utilizarea script-urile disponibile pentru o integrare cât mai ușoară. Și acestea trebuie copiate și importate la rândul lor. Folosind doar clasa VNCViewerSDK se pot apela toate operațiile disponibile pentru viewere: creare, pornire, oprire, distrugere.

În cazul în care se dorește o nouă implementare a clientului, și acest lucru este posibil. În primul rând trebuie scrise un set de metode pe post de callback-uri. Pointerii acestor metode sunt stocați într-un dicționar alături de numele funcțiilor pe care se mapează. În momentul în care un viewer este creat, se transmite și dicționarul care conține funcțiile. În momentul în care este pornit viewer-ul, informația primită trebuie salvată într-un buffer video. De aici mai departe este decizia dezvoltatorului ce face. Pentru mai multe detalii se recomandă consultarea documentației RealVNC și a scripturilor existente.

%\chapter{Conclusions}
 \chapter{Concluzii}
\label{cap:concluzii}
%
%Cuprinde:
%
%\begin{itemize}
% \item un rezumat al contribuțiilor aduse: ce s-a realizat, relativ la ce s-a propus, în ce constă experiența acumulată, care au fost punctele dificile atinse și rezolvată, recomandări pentru alții care abordează tema, la ce este bun ce s-a obținut etc.
% 
% \item a analiză critică a rezultatelor obținute: avantaje, dezavantaje, limitări
% 
% \item o descriere a posibilelor dezvoltări și îmbunătățiri ulterioare
%\end{itemize}
%
%Poate fi organizat pe secțiuni, dacă se dorește.
%
%Se întinde pe aproximativ 1-2 pagini. 

În capitolele precedente au fost detaliate implementarea și problemele apărute pentru dezvoltarea acestui modul. S-a realizat o soluție proprie pentru o necesitate reală. Evident, s-au folosit module și tehnologii existente în implementarea acestui proiect. La fel ca în realizarea prototipului anterior, și acest proiect a ajutat la îmbogățirea cunoștințelor despre diverse subiecte: detalii suplimentare despre funcționarea AMT, tehnici de desenare în Python, scrierea de extensii în limbajul C, etc. Deși nu s-a atins potențialul maxim al avantajelor oferite de AMT, acest proiect și-a atins scopul, și s-a făcut posibilă atât accesarea simultană a mai multor buffere video de la distanță, cât și oferirea acestei funcționalități utilizatorilor de Python 3.5.  

S-au realizat diverse teste cu acest modul. Cel mai interesant test de performanță, cel prezentat și în această lucrare în capitolul dedicat testării, scoate la suprafață informații interesante. Soluția implementată este în momentul de față mai costisitoare din punct de vedere al volumului de date de pe rețea. Deși acesta este un dezavantaj, nu este unul atât de mare. Diferențele dintre RealVNC și acest proiect nu sunt observabile cu ochiul liber. În plus, viteza nu a fost niciodată o prioritate. Important a fost sa obține acces simultan la framebuffere. Lucru care RealVNC nu îl oferă.

Există multe funcționalități ce pot fi adăugate acestui proiect, dar din cele ce sunt strict legate de VNC putem enumera următoarele:
\begin{itemize}
  \item Posibilitatea de a trimite evenimente de input de la tastatură sau mouse, lucru care este deja în dezvoltare
  \item Îmbunătățirea sistemului de management al erorilor
  \item Adăugarea de suport și pentru alte codificări
  \item Posibilitatea de a alege și o variantă mai mică pentru bpp(bits per pixel), facilitând astfel o viteza mai mare
  \item Posibilitatea de a face o captură de ecran, s-a încercat deja și este fezabil
  \item Posibilitatea de detecție mașinilor care s-au blocat sau care au întâmpinat un BSOD( Blue Screen Of Death)
  \item Capacitatea de înregistrare a unui stream video, pentru a fi mai târziu vizionat
\end{itemize}


Ca și concluzie generală, rezultatul este unul satisfăcător deoarece s–a reușit implementarea unei idei într-o manieră nouă pentru a accesa buffer-ul video de la mai multe mașini aflate la distanță.




%\addcontentsline {toc}{chapter}{Bibliography}
\bibliographystyle{IEEEtran}
\bibliography{thesis}%same file name as for .bib

\appendix

%\chapter{Diverse anexe}


%\chapter{Demonstrații matematice detaliate (dacă există)}


\chapter{Cod sursă (parțial)}

% \begin{verbatim}
%  /** Maps are easy to use in Scala. */
% object Maps {
%   val colors = Map("red" -> 0xFF0000,
%                    "turquoise" -> 0x00FFFF,
%                    "black" -> 0x000000,
%                    "orange" -> 0xFF8040,
%                    "brown" -> 0x804000)
%   def main(args: Array[String]) {
%     for (name <- args) println(
%       colors.get(name) match {
%         case Some(code) =>
%           name + " has code: " + code
%         case None =>
%           "Unknown color: " + name
%       }
%     )
%   }
% }
% \end{verbatim}

%\lstset{language=Python, showstringspaces=false, basicstyle=\tiny\ttfamily, linebreak}
\lstset{language=Python,frame=single, basicstyle=\ttfamily\small,  showstringspaces=false, breaklines=true}
\begin{lstlisting}
/* Method used to create a VNC Viewer */
static PyObject *
module_CreateViewer(
    PyObject* self,
    PyObject* args
    )
{
    int retCode;
    char *funcName;
    char *serverName; 
    PLIST_T node;
    Py_ssize_t pos;
    PyObject *dict, *key, *value;
    PyObject *key_str;    
    PyObject *viewer_self;
    VNCViewer *viewer;
    VNCViewerCallbacks vncCallbacks;
    
    UNREFERENCED_PARAMETER(self);
    pos = 0;
    retCode = 0;

    
    // Check arguments
    if (!PyArg_ParseTuple(args, "O!sO", &PyDict_Type, &dict, &serverName, &viewer_self))
    {        
        PyErr_Format(PyExc_TypeError, "You need to pass a dictionary with callback functions");
        return NULL;
    }
   
    // Callback
    memset(&vncCallbacks, 0, sizeof(vncCallbacks));
    InitializeCallbacks(&vncCallbacks);
    
    //Register Callbacks
    while (PyDict_Next(dict, &pos, &key, &value))
    {
        if (!PyCallable_Check(value))
        {
            PyErr_SetString(PyExc_TypeError, "Parameter must be callable");
            return NULL;
        }

        /* Prepare key and check value, then register the callback*/
        key_str = PyObject_Str(key);
        funcName = PyUnicode_AsUTF8(key_str);
        if (!RegisterCallback(funcName, value))
        {
            PyErr_Format(PyExc_TypeError, "Function name not found: [%s]", funcName);
            return NULL;
        }        
    }
   
    // Create Viewer
    viewer = gVNCSdk.vncViewerCreate(&vncCallbacks, sizeof(vncCallbacks), FALSE);
    if (NULL == viewer)
    {   
        LOG_ERROR(0, "Viewer could not be created!");
        retCode = 1;
        goto end_this;
    }

    // You should keep this
    if (!ListInsert(serverName, viewer_self, viewer, vncCallbacks))
    {
        LOG_ERROR(0, "Somehow the insert failed"); 
        retCode = 1;
        goto end_this;
    }

    // Set that Context ;)
    node = ListGetNode(serverName);
    if (NULL == node)
    {
        LOG_ERROR(0, "ListGetNode failed for [%s]. List probably corrupt.", serverName);
        retCode = 1;
        goto end_this;
    }
    gVNCSdk.vncViewerSetContext(node->context->pViewer, node->context);
    
end_this:
    return Py_BuildValue("i", retCode);
}
\end{lstlisting}

\lstset{language=Python,frame=single, basicstyle=\ttfamily\small,  showstringspaces=false, breaklines=true}
\begin{lstlisting}
import numpy as np

class PixelBuffer(object):
    '''    
    Initialized at ServerInitCallback: width, height, desktopName, pServerNativePixelFormat
    Copy desktop updates
    '''
       
    def __init__(self, width_height, desktop_name, rgb_max, rgb_shift, big_endian):                
        self.desktop_name = desktop_name
        self.width, self.height = width_height
        self.red_max, self.green_max, self.blue_max = rgb_max
        self.red_shift, self.green_shift, self. blue_shift = rgb_shift

        # ESENTIAL #
        self.buffer = np.zeros((self.width,self.height))
        if big_endian == 0:
            self.big_endian = False
        else:
            self.big_endian = True
        
        self.was_resized = False 

        # to be used
        self.bits_per_pixel = None
        self.depth = None        
        self.true_colour = None
       

    def __str__(self):       
        pixel_info = '\nPixel Buffer [{}]'.format(self.desktop_name)
        pixel_info += '\n * width [{}] x height [{}]'.format(self.width, self.height)
        pixel_info += '\n * red max   [{}] shift [{}]'.format(self.red_max, self.red_shift)
        pixel_info += '\n * green max [{}] shift [{}]'.format(self.green_max, self.green_shift)
        pixel_info += '\n * blue max  [{}] shift [{}]'.format(self.blue_max, self.blue_shift)
        pixel_info += '\n * big_endian [{}]'.format(self.big_endian)
        pixel_info += '\n * ...'

        return pixel_info

    def update_rectangle(self,tx_ty, bx_by, pixel_data, 
        '''
        Interpret pixel data and update Pixel Buffer submatrix with the new pixel info

        tx_ty: (top left x, top left y)
        bx_by: (bot left x, bot left y)
        pixel_data: submatrix to be interpreted 
        data_len: !!! not used yet !!!
        '''
        top_left_x, top_left_y = tx_ty
        bot_left_x, bot_left_y = bx_by
        
        red = (np.right_shift(pixel_data, self.red_shift) & self.red_max) * 255 / self.red_max
        red = red.astype(np.int32)
        red = np.left_shift(red, 16)

        green = (np.right_shift(pixel_data, self.green_shift) & self.green_max) * 255 / self.green_max
        green = green.astype(np.int32)
        green = np.left_shift(green, 8) 

        blue = (np.right_shift(pixel_data, self.blue_shift) & self.blue_max) * 255 / self.blue_max
        blue = blue.astype(np.int32)

        temp = red + green + blue
        temp = np.fliplr(temp)
        temp = np.rot90(temp)
        self.buffer[top_left_x:bot_left_x , top_left_y:bot_left_y] = temp
        
        
    def fill_rectangle(self,tx_ty, bx_by, pixel_data): 
        '''Fill rectangle at a given pos in Pixel Buffer with a single colour

        tx_ty: (top left x, top left y)
        bx_by: (bot left x, bot left y)
        pixel_data: bytearray to be interpreted 
        '''
        top_left_x, top_left_y = tx_ty
        bot_left_x, bot_left_y = bx_by
        
        if self.big_endian:
            pixel_data = int.from_bytes(pixel_data, byteorder='big')
        else:
            pixel_data = int.from_bytes(pixel_data, byteorder='little')
        
        red = ((pixel_data >> self.red_shift) & self.red_max) * 255 / self.red_max        
        red = int(red) << 16
        green = ((pixel_data >> self.green_shift) & self.green_max) * 255 / self.green_max
        green = int(green) << 8
        blue = ((pixel_data >> self.blue_shift) & self.blue_max) * 255 / self.blue_max
        blue = int(blue) << 0

        self.buffer[top_left_x:bot_left_x , top_left_y:bot_left_y] = red + green + blue

        return True

    def copy_rectangle(destination, source):#NOT TESTED
        '''Copy submatrix within the Pixel Buffer

        destination: (dest->topLeft.x, dest->topLeft.y, dest->bottomRight.x, dest->bottomRight.y)
                     The rectangle on the VNC server desktop that is being
                     updated - that is, the destination of the copy.
        source: (pSource.x, pSource.y)
                The top-left corner of the source of the copy.  The area to
                be copied is the same as the area of *pDestination.
        '''
        dest_top_x, dest_top_y, dest_bot_x, dest_bot_y = destination
        src_x, src_y = source

        x = dest_bot_x - dest_top_x
        y = dest_bot_y - dest_top_y

        self.buffer[dest_top_x:dest_bot_x, dest_top_y:dest_bot_y] = self.buffer[src_x:src_x + x,src_y:src_y + y] 
        
        return True

    def resize_screen(self, new_width_height):        
        self.width, self.height = new_width_height
        self.buffer = np.zeros((self.width,self.height))
        self.was_resized = True
\end{lstlisting}
%\chapter{Articole publicate}



\end{document}
