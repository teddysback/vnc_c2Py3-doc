\appendix

%\chapter{Diverse anexe}


%\chapter{Demonstrații matematice detaliate (dacă există)}


\chapter{Cod sursă (parțial)}

% \begin{verbatim}
%  /** Maps are easy to use in Scala. */
% object Maps {
%   val colors = Map("red" -> 0xFF0000,
%                    "turquoise" -> 0x00FFFF,
%                    "black" -> 0x000000,
%                    "orange" -> 0xFF8040,
%                    "brown" -> 0x804000)
%   def main(args: Array[String]) {
%     for (name <- args) println(
%       colors.get(name) match {
%         case Some(code) =>
%           name + " has code: " + code
%         case None =>
%           "Unknown color: " + name
%       }
%     )
%   }
% }
% \end{verbatim}

%\lstset{language=Python, showstringspaces=false, basicstyle=\tiny\ttfamily, linebreak}
\lstset{language=Python,frame=single, basicstyle=\ttfamily\small,  showstringspaces=false, breaklines=true}
\begin{lstlisting}
/* Method used to create a VNC Viewer */
static PyObject *
module_CreateViewer(
    PyObject* self,
    PyObject* args
    )
{
    int retCode;
    char *funcName;
    char *serverName; 
    PLIST_T node;
    Py_ssize_t pos;
    PyObject *dict, *key, *value;
    PyObject *key_str;    
    PyObject *viewer_self;
    VNCViewer *viewer;
    VNCViewerCallbacks vncCallbacks;
    
    UNREFERENCED_PARAMETER(self);
    pos = 0;
    retCode = 0;

    
    // Check arguments
    if (!PyArg_ParseTuple(args, "O!sO", &PyDict_Type, &dict, &serverName, &viewer_self))
    {        
        PyErr_Format(PyExc_TypeError, "You need to pass a dictionary with callback functions");
        return NULL;
    }
   
    // Callback
    memset(&vncCallbacks, 0, sizeof(vncCallbacks));
    InitializeCallbacks(&vncCallbacks);
    
    //Register Callbacks
    while (PyDict_Next(dict, &pos, &key, &value))
    {
        if (!PyCallable_Check(value))
        {
            PyErr_SetString(PyExc_TypeError, "Parameter must be callable");
            return NULL;
        }

        /* Prepare key and check value, then register the callback*/
        key_str = PyObject_Str(key);
        funcName = PyUnicode_AsUTF8(key_str);
        if (!RegisterCallback(funcName, value))
        {
            PyErr_Format(PyExc_TypeError, "Function name not found: [%s]", funcName);
            return NULL;
        }        
    }
   
    // Create Viewer
    viewer = gVNCSdk.vncViewerCreate(&vncCallbacks, sizeof(vncCallbacks), FALSE);
    if (NULL == viewer)
    {   
        LOG_ERROR(0, "Viewer could not be created!");
        retCode = 1;
        goto end_this;
    }

    // You should keep this
    if (!ListInsert(serverName, viewer_self, viewer, vncCallbacks))
    {
        LOG_ERROR(0, "Somehow the insert failed"); 
        retCode = 1;
        goto end_this;
    }

    // Set that Context ;)
    node = ListGetNode(serverName);
    if (NULL == node)
    {
        LOG_ERROR(0, "ListGetNode failed for [%s]. List probably corrupt.", serverName);
        retCode = 1;
        goto end_this;
    }
    gVNCSdk.vncViewerSetContext(node->context->pViewer, node->context);
    
end_this:
    return Py_BuildValue("i", retCode);
}
\end{lstlisting}

\lstset{language=Python,frame=single, basicstyle=\ttfamily\small,  showstringspaces=false, breaklines=true}
\begin{lstlisting}
import numpy as np

class PixelBuffer(object):
    '''    
    Initialized at ServerInitCallback: width, height, desktopName, pServerNativePixelFormat
    Copy desktop updates
    '''
       
    def __init__(self, width_height, desktop_name, rgb_max, rgb_shift, big_endian):                
        self.desktop_name = desktop_name
        self.width, self.height = width_height
        self.red_max, self.green_max, self.blue_max = rgb_max
        self.red_shift, self.green_shift, self. blue_shift = rgb_shift

        # ESENTIAL #
        self.buffer = np.zeros((self.width,self.height))
        if big_endian == 0:
            self.big_endian = False
        else:
            self.big_endian = True
        
        self.was_resized = False 

        # to be used
        self.bits_per_pixel = None
        self.depth = None        
        self.true_colour = None
       

    def __str__(self):       
        pixel_info = '\nPixel Buffer [{}]'.format(self.desktop_name)
        pixel_info += '\n * width [{}] x height [{}]'.format(self.width, self.height)
        pixel_info += '\n * red max   [{}] shift [{}]'.format(self.red_max, self.red_shift)
        pixel_info += '\n * green max [{}] shift [{}]'.format(self.green_max, self.green_shift)
        pixel_info += '\n * blue max  [{}] shift [{}]'.format(self.blue_max, self.blue_shift)
        pixel_info += '\n * big_endian [{}]'.format(self.big_endian)
        pixel_info += '\n * ...'

        return pixel_info

    def update_rectangle(self,tx_ty, bx_by, pixel_data, 
        '''
        Interpret pixel data and update Pixel Buffer submatrix with the new pixel info

        tx_ty: (top left x, top left y)
        bx_by: (bot left x, bot left y)
        pixel_data: submatrix to be interpreted 
        data_len: !!! not used yet !!!
        '''
        top_left_x, top_left_y = tx_ty
        bot_left_x, bot_left_y = bx_by
        
        red = (np.right_shift(pixel_data, self.red_shift) & self.red_max) * 255 / self.red_max
        red = red.astype(np.int32)
        red = np.left_shift(red, 16)

        green = (np.right_shift(pixel_data, self.green_shift) & self.green_max) * 255 / self.green_max
        green = green.astype(np.int32)
        green = np.left_shift(green, 8) 

        blue = (np.right_shift(pixel_data, self.blue_shift) & self.blue_max) * 255 / self.blue_max
        blue = blue.astype(np.int32)

        temp = red + green + blue
        temp = np.fliplr(temp)
        temp = np.rot90(temp)
        self.buffer[top_left_x:bot_left_x , top_left_y:bot_left_y] = temp
        
        
    def fill_rectangle(self,tx_ty, bx_by, pixel_data): 
        '''Fill rectangle at a given pos in Pixel Buffer with a single colour

        tx_ty: (top left x, top left y)
        bx_by: (bot left x, bot left y)
        pixel_data: bytearray to be interpreted 
        '''
        top_left_x, top_left_y = tx_ty
        bot_left_x, bot_left_y = bx_by
        
        if self.big_endian:
            pixel_data = int.from_bytes(pixel_data, byteorder='big')
        else:
            pixel_data = int.from_bytes(pixel_data, byteorder='little')
        
        red = ((pixel_data >> self.red_shift) & self.red_max) * 255 / self.red_max        
        red = int(red) << 16
        green = ((pixel_data >> self.green_shift) & self.green_max) * 255 / self.green_max
        green = int(green) << 8
        blue = ((pixel_data >> self.blue_shift) & self.blue_max) * 255 / self.blue_max
        blue = int(blue) << 0

        self.buffer[top_left_x:bot_left_x , top_left_y:bot_left_y] = red + green + blue

        return True

    def copy_rectangle(destination, source):#NOT TESTED
        '''Copy submatrix within the Pixel Buffer

        destination: (dest->topLeft.x, dest->topLeft.y, dest->bottomRight.x, dest->bottomRight.y)
                     The rectangle on the VNC server desktop that is being
                     updated - that is, the destination of the copy.
        source: (pSource.x, pSource.y)
                The top-left corner of the source of the copy.  The area to
                be copied is the same as the area of *pDestination.
        '''
        dest_top_x, dest_top_y, dest_bot_x, dest_bot_y = destination
        src_x, src_y = source

        x = dest_bot_x - dest_top_x
        y = dest_bot_y - dest_top_y

        self.buffer[dest_top_x:dest_bot_x, dest_top_y:dest_bot_y] = self.buffer[src_x:src_x + x,src_y:src_y + y] 
        
        return True

    def resize_screen(self, new_width_height):        
        self.width, self.height = new_width_height
        self.buffer = np.zeros((self.width,self.height))
        self.was_resized = True
\end{lstlisting}
%\chapter{Articole publicate}

